\subsection{第一次上班(上)}

“可爱点?”胡子茬啦的大叔老板明显一愣,摸了摸后脑勺,似乎有些不太能理解小女孩觉得可爱的标准是什么,他思考了一会儿,从一大叠放的整整齐齐的被子底下抽出一条巧克力和白色方块相间的棉被,问道,“小姑娘……这条怎么样?”

“唔……有更好看点的吗?”

“那你来看看吧。”

翻了一圈,就只有那条巧克力色和白色方块相间的棉被看起来还舒服些,其他都是款式十分老的大红花,看到那些棉被,感觉就像是回到了十年前,自己住在外婆家的时候,盖的好像就是那种棉被吧……

“算啦,就这条吧。”苏雨晴只能无奈地选择了那唯一一条看起来还算顺眼的。

“好嘞,我给你包起来,其实我家那女儿也喜欢这种颜色哈。”

“七十块钱是吧?”苏雨晴摸了摸口袋里唯一的一张百元钞票,问道。

“对。”

苏雨晴正准备爽快地把钱掏出来,却想了之前老板娘和老虎讨价还价的事情。

每一块钱对于苏雨晴而言都是很重要的,能省一点就尽量省一点,不如就试试看还价吧,能还一点是一点嘛。

苏雨晴看着正在把棉被叠起来装进大红色塑料袋里的老板,清了清嗓子,想了想在电视剧里以及母亲还价时的腔调,酝酿了一会儿,说道:“七十块钱……能便宜点吗?”

结果话一说出口,一点气势都没有,不像是在还价,倒像是在祈求施舍。

而且还因为这是苏雨晴第一次还价,所以说的时候小脸还有些微红,似乎讨价还价是一件很不厚道的事情似的。

“不能便宜了,我这里都是成本价了。”

“那……好吧。”刚才还想着杀价的苏雨晴,此刻早已紧张得将那种想法抛到九霄云外去了,只是有些慌乱地点了点头,见老板没说什么,反倒是自己松了口气。

看别人砍价好像挺轻松的,轻描淡写地就能让价格降下许多,但是偏偏自己来做的时候却这么难呢。

许多事也是如此,看似简单,实际上却并不容易。

而砍价这种东西,首先需要的就是心理素质,一个人想要砍价,就不能让自己太想要那样东西,这样才会外在表现出一种“无所谓”的样子,精明的商人能够分辨你是很想要这样商品,还是购买欲望不强,然后据此来回应你。

所以,苏雨晴的第一次还价以失败告终,但是,失败乃成功之母,只要多尝试,以后也终会掌握这种技巧的呢。

还价这种事情,还是女性比较擅长一些,因为她们更有耐心,情绪表达得也更加清楚强烈。

大多数的男性都不擅长砍价这种事情,也是和他们偏理性的思维方式有关的。

那么,苏雨晴的思维方式到底是偏理性还是偏感性呢?

现在是有些偏向感性,但是思维方式的真正完全成型,还要看以后的发展呢。

说不定哪一天苏雨晴突然觉悟了,变得理性了,可能也会放弃追求那对于现在的她而言十分虚无缥缈的变成女孩子的愿望吧。

买完了棉被之后,一百块钱就只剩下了三十块钱,苏雨晴又去附近的小超市里买了一些必备的生活用品。

比如牙刷、牙膏、茶杯、脸盆、毛巾以及泡沫地垫之类的物件儿,而且都是挑最便宜的买,饶是如此,买完这些东西后三十块钱就只剩下了五块钱。

钱永远都是最好花的东西,苏雨晴摸着口袋里的五块钱,有些感慨地想道。

也正是因为这样,所以才会有那么多人为了钱而前仆后继,去做那些违反道德,甚至毁灭人性的事情吧?

回到小小的家中后,苏雨晴再次忙碌了起来。

将脸盆之类的洗漱用具都放在了一张五块钱买来的小板凳上,之所以不放在卫生间里,是因为卫生间实在是太小了,如果把这些都放进去,会让那个狭小的空间更加狭窄,可能连转个身都会很困难呢。

廉价的泡沫地垫被苏雨晴铺在了地上,虽然上面所涂的颜料十分劣质,但是当她将所有的泡沫地垫都铺上的时候,仍然觉得十分的满足。

铺上了地垫的地方就算是“卧室”范围,进“卧室”需要拖鞋,苏雨晴可以赤着脚站在软而温暖的泡沫地垫上,也可以在上面肆意地打滚,从某种意义上来说,这相当于扩展了这个小房间的空间——可利用的空间。

忙完后,时间已经不早了,苏雨晴把闹钟放在了桌上,然后设定好了响铃的时间。

老板娘说上班的时间是早上六点,虽说这里距离面馆很近,但毕竟是刚刚开始上班,迟到可会给别人留下不好的印象的呢,所以苏雨晴把时间调在了五点二十分,这个时间起床就肯定没问题了,足足四十分钟,时间是十分充裕的呢。

卫生间里当然不像苏雨晴家有热水器这种在这个时候算是比较高端的设施,放出来的水自然也是冰凉冰凉的,甚至比外面的温度还要冷上三分。

钱已经不多了,如果想要买一个热水瓶和热水壶的话,可能会把剩下的钱都给花光,无论如何,身边都得留一些钱用以防意外,特别是独自一人生活,没有钱就没有安全感,所以虽然水冰了点,但苏雨晴还是咬牙忍下了,没有去买热水瓶和电热水壶。

“呼……”刚用冰水洗过脚的苏雨晴浑身都有些发抖,她哆嗦着走到了床边,掀开刚买来的巧克力色的格子和白色的格子相间的棉被,一下子就钻了进去,紧紧地裹着棉被,蜷缩在那张小小的床上,就像是一只怕冷的小猫一般。

昨天苏雨晴根本就没有睡好,今天又累了一天,虽然时间才晚上八点,但她已经觉得很困了,一只手刚把床头连接着灯的开关关掉,就陷入了沉睡之中。

有人说,如果一个人在白天很累的话,那么她在晚上就不会做梦,因为大脑也已经疲惫得不想思考了。

没有梦的睡眠也是最让人觉得舒服的睡眠,有些人甚至会因为晚上做了梦,早上起来的时候感觉更累呢。

苏雨晴今天一整个晚上都没有做梦,一直到耳边传来了清脆的鸟啼声,才从那让她感觉十分舒服的梦境中缓缓地醒过来。

在叫的并不是鸟儿,只是苏雨晴的闹钟而已。

苏雨晴轻轻地将闹钟关上,睁大着眼睛,看着此时还一片漆黑的房间。

窗外的太阳才刚刚从地平线处升起,虽然有些许的光亮,但也有限的可怜。

苏雨晴就这样盯着漆黑的天花板发了一会儿呆,直到看到天花板上那一抹骄阳的微光后才回过神来。

“姆……已经是早上了呀……”苏雨晴揉了揉太阳穴,觉得还想再睡一会儿,但是闹钟的时针已经走到了四十的位置,再不起床的话可就要迟到了呢。

虽说是初春,但是早上起来的难度并不比严冬时要小多少,苏雨晴几乎是鼓起勇气才将棉被掀开,像是生怕自己反悔一样,一下子就从床上跳了下来,站在软软的泡沫地垫上伸了一个大大的懒腰,这才将衣服一件一件地套上。

有时候洗脸的水很冰也是一件好事,最起码能让苏雨晴一下子就清醒过来。

卫生间虽然小,但是房东竟然在里面装了一面镜子,苏雨晴在洗漱的时候就盯着镜子里的自己看,而且在许多能反光的地方,苏雨晴都会下意识地看看自己,大概是生怕仪容不够整齐,或者头发没有打理好吧。

从某种意义上来说,这也算是女孩子的习惯了吧?

只是今天并没有头发可打理,苏雨晴揉了揉自己那短得可怜的平头,不断地祈祷着希望它能快点长起来,没有了父母的约束,苏雨晴也就更想要留一个像普通女孩子那样的长发了。

对于普通女孩子而言,留长发简直是再正常不过的事情,但对于苏雨晴来说,却是一个一直想要实现的愿望……

虽然头发很短,但苏雨晴还是习惯性地用一把可爱的小木梳轻轻地梳了梳自己的头发,梳子在头皮上轻轻拂过的感觉,让她觉得非常的舒服,不比在出了一身大汗后洗个冷水澡的感觉差。

“唔……长了颗痘痘……”苏雨晴轻轻地拨弄着额头上一个小小的痘包,不知道是该挤掉还是该留着不管它,因为她还从来没有长过痘痘呢。

“算了……就这样好了……”苏雨晴把鸭舌帽往下拉了一点,把那个小小的痘痘盖住,再看了看镜子里的自己,这才满意地点了点头。

最后在镜子里看了看自己的着装打扮没有问题之后,苏雨晴才戴上帽子走出了门。

她依然穿着男装,这种小号的男装穿在她的身上,总会让人有一种这是特意为女性而设计的男装的感觉呢。

离开了父母,独自一人生活在小城市里,也独自一人睡了一个晚上,苏雨晴对于自己昨天晚上竟然能睡得那么好都觉得有些奇怪,因为她是一个比较认床的人,在其他地方总是睡不安稳,特别是一个人睡……

“或许是因为昨天晚上太累了吧……”苏雨晴压了压帽檐,贴着墙根慢慢地走着,吹着那带着些许芳草气息的冷风,走到了面馆前。

苏雨晴抬起头,看了看面馆的招牌,既然要好好工作了,总不能连工作的店叫什么名字都不知道吧?

“无名面馆……”

这是一个看起来很有特色,实际上大概是店主想不到取什么名字而取的店名吧……

“好随便的名字……”苏雨晴小声地自语着,走进了刚开门没多久的面馆里。

……
