\subsection{独自一人,不知方向}

“唔……唔?”苏雨晴有些迷糊地抬起头来,看到那排着队缓缓下车的人群,这才猛然反应过来。

从杭州到小城市的这班火车没有停靠站,属于直达车,如果火车开门了,就代表目的地已经到了。

五个多小时的旅程有些漫长,时间也已经是下午了,天空中依然笼罩着阴云,因为没有太阳,所以看不出到底有没有到傍晚,只是让苏雨晴觉得光线比中午的时候好像黯淡了许多。

那个叫醒苏雨晴的人已经汇入了人群之中,仅仅只是听到过他说过一句话的苏雨晴,肯定是无法辨别到底是哪个好心人把她叫醒了。

这样的善意的提醒和帮助其实在身边有许多,只是因为实在微乎其微,所以总是被人们经常地忽略而已。

当然,那些微小的恶意也着实不少,一切的事物,都有着其两面性,不是吗?

苏雨晴轻轻揉了揉还有些迷糊的脑袋,拖着自己的行李箱跟着人流走出了火车,站在了那有些空旷的火车站的站台上。

一阵阴冷的风吹过,似乎在阴笑着钻入了苏雨晴的衣服里,那种冰冷的感觉让苏雨晴那娇弱的身体一阵颤抖,那刚才还尚存着的几分睡意,一下子就烟消云散了。

苏雨晴拉了拉鸭舌帽的帽檐,在确定了自己的帽子还好好地戴在她的头上,遮挡着那在她看起来可耻而又可笑的平头短发。

她的一只手拉着行李箱,而另一只手却总是放在鸭舌帽上,生怕那顶帽子因为什么意外而掉落,仿佛她的平头一露出来,就会受到所有人的耻笑一样。

站台上的人流并不多,算不上特别热闹,在冷风的吹拂下,倒是显得有几分冷清。

小城市的火车站站台没有设在地下,而是直接建造在地面上,并且没有太多的遮挡措施,更没有什么窗户之类的东西,它就像是一个巨大的“棚”,除了把头顶的天空笼罩住外,其他地方根本就是完全暴露在空气中的。

不像杭州的火车站,四面都是围住的,人在里面,能够感受到温暖,小城市的火车站则不同,在这个初开春的时节,只能让人感到冰冷……

初春的温度不比冬天高多少,有时候甚至还会下几场冬雪,苏雨晴虽然穿了毛衣和一件厚厚的棉袄,也仍然冷得发颤。

苏雨晴是第一次来小城市的火车站,这里的环境对于她而言可以说是绝对的陌生,而且四周没有一个她认识的人,也没有本来总是在她身边的父母和亲人……

举目四顾,只有她一个人。

那些其他的乘客,在此刻的苏雨晴眼里,就像是游戏中没有温度的,只是一串冰冷数据的 NPC。

苏雨晴在原地发呆了好一会儿,终究还是一咬牙,跟着人群队伍的尾巴走了上去,最起码这样还会热闹一些,哪怕那些人和她一点关系都没有。

她可不想独自一人待在这里品尝孤独和寂寞,虽然这是以后必然要品尝到的东西,但是能让心中多些安慰,就尽量地多一些吧,最起码,那样会让她觉得好受一些。

跟着人群走出了小城市这个简陋的火车站,苏雨晴再次停下了脚步,她看了看左边,又看了看右边,还看了看前方的道路,不知道自己该走哪个方向。

就像是迷途的候鸟一样,苏雨晴的心中除了迷茫外,还有焦虑和沮丧。

十五岁,正是孩子心中产生浓厚独立意识的年岁,但是很多孩子都无法做到真正的独立,他们总是会下意识地想要依靠着谁,而当真正一个人走在外面,无依无靠的时候,那种惶恐,那种害怕,就会一股脑地涌上心头。

该……朝哪里走?

苏雨晴朝每个方向都迈了半步,但每次都收了回来,在原地打着转,像是从父母身边走失而等着父母来找回她的孩子一样。

耳边,是有些嘈杂的吆喝声,还有一些胡子茬啦的大叔和一脸皱纹的妇女走过苏雨晴的身旁,向她询问着,但都以苏雨晴的沉默而没有了后续。

“来来,这位兄弟,去哪里?”一位大叔拦住了一个不修边幅的年轻人,问道。

“去白石路,多少钱?”不修边幅的年轻男子十分熟练地问道,显然并不是第一次坐这种私人摩托车了。

“十块钱,比打的便宜!”

“行吧,走。”不修边幅的男子也没有多废话,直接跨上了中年大叔的摩托车,说道。

“好嘞,您坐稳。”

那位坐在摩托车上的男子看到苏雨晴的时候愣了愣,似乎想说什么,但是却早已被摩托车带远了。

摩托车发出震耳的轰鸣声,在一片烟尘中远去了。

“要住宿吗?二十块钱一个晚上,还包晚餐啊。”

“小姑娘,要住宿吗?”

“……”

“来来来,刚出炉的驴打滚勒,香甜软糯不黏牙!”

“糖葫芦,糖葫芦勒,又大又甜的糖葫芦勒!”

诸如此类的声音此起彼伏,不绝于耳。

小城市的火车站门口也是那样的富有乡土气息,好像这里不是什么火车站,只是一个大型的集市而已。

苏雨晴站在原地已经快有半个小时了,不知道的人或许以为她是在等人吧。

“……不要再去想她们了,一切……都靠我自己。”苏雨晴轻轻地晃了晃脑袋,将心底升起的软弱压了下去,再次看了看四周,最后总算选定了方向,不急不缓地朝前走去。

之所以不急不缓,其实是因为苏雨晴也不知道自己要去哪里,没有目标,自然就不会焦躁了嘛。

这是苏雨晴第一次独自外出,她的思绪还有些混乱,她需要时间来将许多事情在脑海中一点一点地整理出来。

“咕——”苏雨晴正微皱着眉头思考着以后的事情时,肚子突然发出了抗议,而且还不是一声,是连续的好几声,这让苏雨晴的小脸有些微红,她低着头悄悄地看了看四周,发现没有人看着自己,这才稍稍松了口气。

苏雨晴总是微微地低着头,像是在看路,其实是在故意不把自己的脸完整地展现出来,所以她视线中总是看不到人的脸,看到的,往往都是人的两条腿,两只脚……

是自卑吗,或许是吧,或许,还有些出于保护自己的想法在内吧。

只有表现得不引人注目,才不会惹上太多的麻烦呢,这是苏雨晴在学校里总结出来的经验,她曾经因为长得太像女孩子,而常常受男生们欺负,女生们排挤……

更何况,是在这样一个陌生的城市里,还是低调一些比较好吧。

“唔……肚子好饿……”苏雨晴轻轻地揉着肚子,她今天一天都没有吃东西,甚至连水都没有喝一口,此刻走在这条开着许多小吃店和饭店的街上,就愈发地觉得难受了。

一天没进食,也让苏雨晴的身子有些乏力,这是低血糖所引发的症状。

解决的办法就是吃点东西。

苏雨晴的身上带着一千五百块钱,听起来很多,但是未来还一点方向都没有呢,自然得要提前节省起来。

只是……

“第一次来到这个城市的第一餐,就稍微吃一点好的吧?也算是犒劳一下自己?”苏雨晴喃喃自语地说着,离开了那家卖馒头的店铺,走到了一个小小的快餐店外,快餐店外摆着一张桌子,桌子上放着一个透明的玻璃“方块”,里面摆着各种各样的佐料和配料,一旁的小凳子上放着一个圆圆的木桶,看起来像是卖糯米饭的地方。

苏雨晴揉了揉肚子,又看了看有些脏乱的快餐店,有些轻微洁癖的她突然就不想走进去了,但是眼尖的老板娘却已经迎了上来,问道:“小姑娘,要吃点什么?”

正准备掉头离开的苏雨晴也只好停下了脚步,她的生活阅历还很少,对于这样老板特别热情的情况,实在是不知道该如何去应付。

她只好有些敷衍地指了指木桶,问道:“这是什么?”

“哦~这个啊,这个叫嵌糕,你是外乡人吧?”

“嗯……”

“小姑娘从哪里来的呀?”

“杭州……”苏雨晴有些架不住老板娘的热情了,但又不好意思就这样转身离开。

算了,还是买点东西吧。

“哦~杭州应该没有嵌糕卖吧,尝尝看吧,味道很好的哦。”

“唔……那就……来一块吧。”苏雨晴对所谓的“嵌糕”也产生了一点点好奇心。

只见老板娘从木桶里拿出一大块热乎乎的年糕,在菜板上揉搓滚平,而后抬头问道:“小姑娘,要什么料?”

“料……?”

“嗯,看这里,要些什么?大排啊、大肠啊、榨菜啊什么的。”

“……就大排吧。”

“好。”

老板娘麻利地将大排切碎,然后放入了年糕里,又加上一些汤啊、油啊以及一些豆芽什么的东西,然后把年糕包起来,样子就像是一个大水饺。

“多少钱……”

“两块。”

苏雨晴付了钱,单手抓着年糕,单手拉着行李箱,再次上路了。

嵌糕的味道很好,而且量也非常足,用来作为一顿充饥的食物,是非常合适的。

苏雨晴的胃口很小,她只吃了半个,就已经觉得很饱了。

吃饱了之后,身上也恢复了些许的力气,那些不愉快的东西也暂时地被苏雨晴忘记,她理清了自己的思绪,开始朝着想要去的地方走去。

……
