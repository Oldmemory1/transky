\subsection{那朦胧的雨}

不大不小的雨,淅淅沥沥地下着。

面馆的透明玻璃门上都淋满了雨水,雨水或是成一条直线,或是七歪八扭地向下滑落,落在地上,和其他先一步到达的“雨滴”汇聚在一起,形成一个个小小的浅浅的水坑。

每到下雨天时,面馆的生意总会差上许多,今天也不例外,整整一个上午,也只不过来了五个客人而已。

李叔叔和张阿姨坐在椅子上悠闲地聊着天,干着活,还时不时地看一眼仍然在沉睡当中的苏雨晴。

“轰隆隆隆——”阴沉而有些黑暗的天空中突然划过一道刺目的闪电,一声震耳欲聋的声响传入人们的耳中,这一道惊雷,让天空都仿佛有些颤抖。

“哗啦啦啦——”原本不大不小的雨在这声惊雷响过之后,又变得更大了几分,从中雨变成了大雨。

大雨倾盆,雨点重重地敲击在雨棚上,发出炒豆子般的声响。

这一声惊雷也将苏雨晴梦中那黑暗的世界劈碎了,她终于从梦中挣脱出来,缓缓地睁开了眼睛。

长长的睫毛轻轻地抖动着,上面还挂着几颗晶莹的泪珠,那是身体为了保护刚睁开的眼睛而分泌的液体。

“醒啦?”李叔叔笑着问道,“饿不饿,要不要吃点东西?”

苏雨晴有些不好意思地摸了摸肚子,轻轻地点了点头,道:“嗯……那个……抱歉……早上……睡着了……”

“没事,反正今天也很空呢。”张阿姨微笑着说道,“对了,小晴,我和你李叔叔商量了一下,以后你的上班时间改一下,早上九点来店里就行,然后每个星期的星期一都给你放假,正好可以给你早上多睡会儿,年轻人嘛,也不能总是待在店里,所以那放假的一天你也可以多出去走走哦。”

“诶?——”苏雨晴面露喜色,有些不太敢相信地问道,“真的吗?”

“当然是真的呀。”

“太好了!谢谢老板和老板娘!”

对于苏雨晴而言,这确实是一个好消息,最起码每天可以睡会儿懒觉,不用晚上睡得迟,早上又起得早了,而且每个星期的放假时间也可以让她放松一下,哪怕在家里睡一整天的觉也是不错的呢。

“生活就算再艰难,但只要努力下去,就一定会好起来的呢……”苏雨晴在心中默默地想道。

不过,有些事情,还是不要过早地下定论才是,毕竟这可不是一个善有善报,恶有恶报的,宛如宗教典籍一样的世界呢。

睡了一觉后,苏雨晴感觉精神也好了许多,胸口不再闷得难受了,脑袋虽然还有些迷糊,但已经比较清醒了。

她深吸了一口因为大雨而变得有些沉闷的空气,第一次觉得能够这样畅快地呼吸是那样让人感到幸福的事情。

“哟,今天没人嘛。”一个年轻男人的声音从门口传来,苏雨晴下意识地抬头看去,好像是昨天来吃过面的食客,他不修边幅的样子给人的印象会稍微深一些,不至于第二天就完全忘记了。

“老板,来碗大排面,加一,加油渣,再加个荷包蛋。”不修边幅的年轻男人收了伞,轻轻地抖了抖,然后走进了面馆里。

他的声音有些沙哑,但也不算特别低沉;有些磁性,但也不算特别多;有些特点,但也不算很特别。

只是芸芸众生中的一个普通人吧。

隐约间,苏雨晴感觉他的声音似乎和火车上叫醒自己的人的声音十分相似,但又无法肯定,因为有这种声线的人虽然不是很普遍,但也实在不能算少。

或许只是苏雨晴多想了,这世界上又哪有那么多巧合呢。

“好嘞,您稍等。”李老板赶紧站起身来,捞了一把面就走进了厨房里。

不修边幅的男人轻轻地哼着有些跑调的小曲儿,看起来不仅没有被下雨天影响心情,反倒觉得心情还不错的样子。

大排面很快就上来了,李老板随后给苏雨晴炒了一份炸酱面,也端到了她的面前。

苏雨晴小口小口地吃着,一点一点地品嚼着面条的味道,这是她从小养成的习惯,或许也和家规有关吧,吃饭的时候必须要细嚼慢咽。

让苏雨晴有些奇怪的是,那个看起来胡子茬啦,十分粗犷而不修边幅的男人吃面的动作竟然也这么轻柔,他也如苏雨晴这样慢慢地一口一口地吃着,而不像大多数男人那样把面吃得“稀里哗啦”地响。

“老板,你这里的面,味道真不错,好像是加了什么特殊的酱料吧?”年轻的男人问道。

“呵呵!当然,里面加了特制的酱料,有花生,有芝麻,还有小城市的特产熔岩辣椒,这种辣椒的籽很辣,但是如果把籽去掉味道就又会像甜椒一样,最多只有些许的微辣,而且这种辣椒特别的香。”

“哦~难怪,我说我没放辣椒怎么有些微辣的感觉呢。”年轻男人轻轻地点了点头,端起面喝了几口汤,咂了咂嘴,再次赞叹道,“这汤的味道更好啊……”

“呵呵,小伙子,我看你和其他的年轻人不一样,其他的小伙子觉得好吃,都会吃得很快,我看你,反倒是越吃越慢嘛?”听到有人称赞自己的厨艺,李老板也觉得心情愉悦,免不了就要多聊几句了。

“好吃的东西,自然要慢慢吃啊,牛嚼牡丹,那可是相当浪费的行为。”

“哈哈,小伙子,说话一套一套的嘛,文化人吧?”

“嗯,读了几年书。”

“现在上大学还是?”

“没读,高中毕业后我就开始工作了。”年轻男人叹了口气,却看不出来有多忧愁,反倒是微笑道,“工作,难找啊。”

“是啊,不过,看起来,你好像不是很担心吧?”

“哈哈,有什么好担心的,车到山前必有路嘛,这样忙忙碌碌,如同浮萍般飘摇不定的人生,有时候还是挺有意思的呢!”

“哈,小伙子,不是一般人啊,这境界,比我都高了。”

“世界上哪有一般人,每一个人都不一般,哪怕是一样的普通,都有不一样的平凡呢!”年轻男人咬了一口大排,一边咀嚼着,一边含糊不清地说道。

苏雨晴在一旁听着,总觉得这个最多二十来岁的年轻人懂的东西真的很多,他说的话,似乎也总是蕴含着些许的哲理。

不过,能拥有这样丰富的人生阅历,所经受过的困难和挫折也一定不少吧?

“小伙子现在做什么工作?我看你上班挺晚的,不会是白领吧?”

小城市里也有写字楼,也有白领,只是相较大城市里的白领来说,要显得更普通平凡一些,工资也更低一点,但仍然是许多人想要去做的工作。

只是在这个年代要成为白领并不是一件容易的事呢,事实上,在这个年代,还有许多人将白领和有钱人划上等号。

“当然不是。”年轻男人笑着摇了摇头,道,“我只是偶尔打打零工,偶尔写写故事的人而已。”

“写写故事?”李老板睁大了眼睛,说话时也不由地用上了敬语,“难道您是一位作家?”

“哈哈……”年轻男人轻笑着摇了摇头,道,“当然不是,我只是一个写故事的人而已。”年轻男人说着,将剩下的一点面塞进嘴里,又喝了一口汤,问道,“老板,多少钱?”

“五块钱。”

“给。”年轻男人将一张五块钱的纸钞放在桌上,然后有些洒脱地转身离去。

“蓬——”他撑开那顶白色的雨伞,走进了大雨中,雨点落在他的伞上,发出清脆的声响。

“下次常来啊——”李老板站在门口大喊道。

“当然。”年轻男人背对着李老板轻轻地挥了挥手,身影渐渐远去,直到消失在这朦胧的瓢泼大雨之中。

苏雨晴收回了注视着那个年轻男子的目光,继续吃起了自己的炸酱面,细细地品嚼着老板所说的味道,有点花生和着芝麻的香味,还有辣椒微辣的香味……

混合在一起。

虽然苏雨晴吃东西一直都习惯细嚼慢咽,但这还是她第一次真正地用心去感受食物的味道,想象着炒菜的时候用了几分火候,放了多少的盐和味精,又翻炒了几次……

雨一直下着,今天的天空就没有亮堂起来过,那路边的路灯也一直散发着昏暗的光芒为街道照明,可见今天的能见度确实特别的低呢。

雨,总是让人有一种伤感的感觉,不由自主地会想到那些伤心的往事,归根结底,还是因为下雨的形式有些像落泪,而那阴沉的天空也像自己伤心时的心情,正是因为这样的共鸣,才会让人总是回想起那些曾经的伤心往事吧。

“不知道爸爸妈妈有没有找我呢,他们会去公安局立案吗?”

“还是说他们会反倒觉得高兴,然后重新生下一个孩子用来替代我呢?”

“如果他们找不到我,会焦急吗?会紧张吗?会担心吗?”

“他们会去哪里找我呢……”

“如果真的找不到,他们,会放弃吗?”

“不知道他们现在过得怎么样……”

“不知道他们晚上会不会因为我而睡不着……”

“如果他们放弃了我后,会把我房间里的东西都扔掉吗?”

“如果他们真的生了孩子,希望那个孩子是一个真正的……男孩儿吧……”

苏雨晴仰望着那阴沉的天空,在心中默默地想着。

