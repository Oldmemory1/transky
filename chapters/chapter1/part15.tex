\subsection{螺内酯的副作用}

苏雨晴是被尿憋醒的。

黑暗的梦中空间迫于身体的压力无法再将她困在梦境中。

苏雨晴睁开眼睛,看着黑漆漆的天花板,窗外唯一的光源就是那昏暗的路灯。

淅淅沥沥的小雨还在下着,根本没有一点停下来的意思。

这个时候就体现出在自己的房间里有卫生间的好处了。

这么冰冷的夜晚,还要穿上衣服跑出去上厕所,那也太麻烦了,一趟下来基本上睡意都没有了,然后等躺到床上重新睡着的时候,天都快要亮了。

苏雨晴伸手打开床头的开关,灯亮了起来,在这漆黑的夜中格外的显眼。

在一个漆黑的夜中,还是待在只有她一人的房间里,总会让苏雨晴觉得离开被窝就会被隐藏在黑暗里的怪物给吃掉一样,只有开了灯才会觉得安心一些。

因为光明可以驱散黑暗嘛。

放在桌上的闹钟时针指在三的位置,现在是凌晨三点多一点,到五点二十分起床,还有两个小时左右的时间可以睡。

苏雨晴打了一个大大的哈欠,对于还十分困倦的人而言,哪怕只剩下十分钟恐怕也会义无反顾地睡着吧。

“咔嗒。”

灯,关了。

苏雨晴抱着温暖的被窝调整了一个舒服的睡姿,再一次沉入了那无垠的黑暗之中,并不是没有梦的睡眠,只是这个梦的世界漆黑一片而已。

然后……苏雨晴再一次被尿憋醒了。

接下来的两个小时里,苏雨晴几乎是每半个小时就醒来一次上厕所,频繁的起床让苏雨晴的睡意很快就消散了,最后干脆就躺在床上,睁着眼睛等着闹钟响起。

清脆的鸟鸣声响起,苏雨晴轻轻地拍下闹钟,铃声停止了,已经因为起来数次上厕所而十分清醒的苏雨晴叹了口气,从床上爬了起来。

结果又一个晚上都没有睡好呐……

苏雨晴对着镜子仔细地刷着牙齿,在感觉到牙齿看起来已经很干净了以后,才漱了几口水吐掉。

额头上的痘痘虽然没有增多,但是有几个明显变大了,这让苏雨晴相当地苦恼,但除了更仔细地把脸洗干净外,也没有别的办法了。

这才刚洗漱完,苏雨晴又感觉到了一阵尿意。

她有些无奈地对着卫生间里小小的蹲坑,对齐那个小小的黑漆漆洞,有些无力地将透明又略有些偏黄的液体挥洒了出去。

明明已经上了这么多次厕所了,每一次还都能放出这么多液体,实在是让苏雨晴觉得有些奇怪。

上完厕所的苏雨晴整理好行装,将鸭舌帽戴在头上,正准备出发,却突然想起了一件事。

“等等……今天好像……九点钟上班?”

有什么事情比早上被闹钟吵醒,却发现今天可以晚点起床更让人高兴呢?

时间是五点四十多分,苏雨晴想要睡觉的话还有很长时间可以睡,不过一想到今天自己有些奇怪的身体状况,她顿时又不想躺到床上去了。

不然刚睡着就被尿憋醒,那种事情实在是太过痛苦了呢……

“到底是怎么回事……昨天晚上也没有喝很多水呀……”苏雨晴有些头疼,“难道是……药的作用?唔……记得螺内酯……看看说明书……”

苏雨晴从抽屉里翻出一盒未拆封的螺内酯,把里面的说明书抽了出来,摊开放在桌上。

螺内酯是一种利尿药,且含有抗雄的成分,虽然是一种低效的利尿药,但如果服用的量过多,效果还是非常明显的,特别是在晚上。

而根据个人体质的不同,效果也会不同,有些人吃了以后顶多就是晚上起来上一次厕所,而有些人吃了以后,则会不停地上厕所。

面对这种情况,应该酌情减少药量,等到身体对药物产生些许免疫力之后再逐步增加。

不然的话,就会出现苏雨晴的这种情况。

不过,苏雨晴并不知道解决的办法,也就只能继续这样吃着了,一般来说,身体都会产生抗体的,到时候就不会这样想要多次上厕所,当然了……这种事情对肾的负荷也是相当大的。

苏雨晴有些头疼地把说明书放回了抽屉里,她也想过减少药量,到是一想到自己正处在发育期,如果不控制好的话,很有可能会向她所不希望的方向发展,所以最后还是一咬牙,忍了下来。

这些,也只是在通往梦想的旅途中,一样微不足道的代价而已。

雨,不知道什么时候停了。

笼罩在天空中的乌云也渐渐散去,久违的阳光穿过薄薄的云层照耀在大地上,虽然它只是一轮初升的骄阳,但是却让苏雨晴感觉那样的温暖。

苏雨晴打开窗,深吸了一口气,清冷的风从窗外悄悄地溜进来,绕着苏雨晴的身子转一圈,又偷偷地跑了出去。

一场春雨过后,天气也有些暖和了起来,或许再下几场雨,天气就彻底地温暖起来了吧。

现在才有点春天的感觉嘛。

苏雨晴深吸了一口气,空气很清新,有着鲜嫩的芳草的香味和些许若隐若无的花香。

在这样一间小屋子里,娱乐的东西也有限,苏雨晴没有把手机带出来,不然好歹能玩玩贪吃蛇和俄罗斯方块。

当然,也不是没有能够自娱自乐的事情,比如可以拿白纸折一些纸飞机或者千纸鹤什么的东西;又或者在记事本上随便地涂涂写写;再或者也可以看看小说。

苏雨晴从初一的时候就开始看小说,对于小说这种东西一直都很迷恋,只是没有初一那时候那么狂热而已。

但是从她打算离家出走,一个人生活时,行李里还装上几本小说来看,她对于小说确实是相当喜爱的呢。

带来的小说并不多,苏雨晴只是挑了几本她怎么也看不厌的,苏雨晴从抽屉里把几本小说翻了出来,感觉自己好像没什么兴趣去看一个完整的故事,最后还是选了鲁迅的散文杂集,随意地翻开一页,悠闲地看了起来。

鲁迅所说的一些话总是一针见血,让人看了就觉得心中畅快至极,而且他也是中国近代史中最擅长骂人的文学家之一,总是拐着弯子抨击一种人或一种群体,而且还不带一个脏字。

虽然有很多内容苏雨晴都读不懂,或者一知半解,但是那种意境,那种水平,却是苏雨晴所向往,所崇拜的。

不为什么……只因为……看起来感觉很厉害。

孩子的想法,有时候就是这么简单。

苏雨晴的双眼停留在了这一页纸中的一行字上。

【走上人生的旅途罢,前途很远,也很暗,然而不要怕,不怕的人面前才有路。】

之所以一下子就注意到了这句话,是因为在这句话的下面划了一个小小的波浪线,是苏雨晴曾经在读的时候做下的笔记。

在当时,这句话给了苏雨晴很大的触动,或许,也正是因为这句话,才驱使着苏雨晴走上了这条路,虽然不能说它给了苏雨晴多大的勇气,但是最起码将那扇紧闭着的门打开了一条小小的缝隙。

苏雨晴的脸上露出一丝苦笑,她又想到了那些对于她而言不太好的回忆……

她又想到了自己的父母,又想到了那天曾发生过的事情。

她也没有想到过,原来事情会来得这么快,这么猛烈,有时候苏雨晴闭上眼睛,再睁开,还会有一种自己还在学校里好好上课的错觉。

苏雨晴不知道自己这么做到底是对还是错,她只知道自己想要这么做,或许是任性吧,她这个在其他事情上都十分听话的乖孩子,却偏偏在这种事情上任性呢。

或者说,苏雨晴其实也是一个倔强的人吧,她所决定的事情,总是难以改变,哪怕有所动摇,也会再坚定起来,努力地踉跄着朝前走去。

风,轻轻地吹着,吹起丝丝缕缕淡淡的愁绪。

苏雨晴觉得自己似乎比以前更多愁善感了,哪怕只是一个人简单地坐着,都能想到那么多乱七八糟的事情。

再回过神来的时候,时间已经是八点五十分了。

神游的时候,时间总是过得异常的快呢。

“唔……糟了!”苏雨晴慌忙地拿起钥匙,冲出了房间,房门被“砰”的一声用力地关上了。

苏雨晴一路小跑地来到面馆,看了一眼那个挂在面馆上面的大时钟,在确定自己没有迟到后,才松了口气。

其实面馆距离苏雨晴的家并不远,一路小跑也就三五分钟的时间而已。

只是这种剧烈运动实在是让苏雨晴感到很吃力,哪怕跑的速度并不快,哪怕跑距离并不远,而且还走走停停的,但是当到达目的地后,她都感觉有些喘不过气来了。

“呼……呼……哈……呼……”苏雨晴将双手撑在膝盖上,就像是个破风箱一样大口地喘着气。

“小晴啊,怎么了?跑得很累嘛。”张阿姨问道。

“呼……我怕、迟到……呼……就……跑……跑过来了……呼……好累……”苏雨晴脚步虚浮地找了个位置坐了下来,足足休息了有十分钟才缓过劲来。

苏雨晴舔了舔嘴唇,感觉喉咙干得快冒烟了,之前就因为总是上厕所而身体有些缺水,现在更是脱水得厉害,因为出来的匆忙,连茶杯都没带,只好拿了一个一次性杯子倒了一杯开水,顾不得烫就直接往嘴里倒。

“咿呀!好烫!”

“慢点喝,不着急。”张阿姨柔声说道。

“嗯……”苏雨晴有些脸红地点了点头,就这样盯着杯子看了足足三分钟,待确定真的凉了一些之后,才将它一饮而尽。

……
