\subsection{找不到答案的梦}

俗话说,日有所思,夜有所想,虽然显得通俗了点,但是梦这种东西,大体上来说就是这个意思。

就是人类本身的精神活动的产物,只是因为人在睡眠时大脑不完全清醒后的某种“幻想”和“错觉”。

这种幻想其实和平时在大脑里想一些事情是一样的,只是不如清醒的时候清晰而已,毕竟梦这种东西,是无意识的嘛。

但不可否认,梦,有时候也能反映一个人的心理状态,只是很多人都不明白自己的梦要向自己表达什么而已。

一般来说,一个人晚上会做好多个梦,等到第二天起来几乎全都忘光了,就算有记得的,也是最后快醒来的时候做的梦,而且那种梦如果不记录下来,也会很快在记忆中消失。

除了这种做了就忘的梦之外,还有一种梦会反复地做,不仅场景一模一样,甚至就连发生的事情都一模一样,即使是在清醒的时候也能清晰地记得梦中所发生的情节……

在很多人眼里,这样的梦更像是寓言,预示着未来将会发生的某些事情。

还有一种说法是,当一个人特别关注某一样事物时,就会做那种重复情节的梦。

就好像快高考的学生总会在高考的前几个晚上梦见自己正坐在考场里写着试卷的梦是一样的吧。

而苏雨晴,也再一次地来到了这个梦境中。

天空中挂着一轮被淡淡的云层遮掩,有些若隐若现的弯月,除了月亮外,其他的一颗星辰都看不见,整片天空基本都是黑暗的。

路灯很昏暗,时不时地还会闪烁两下,发出几声电火花呲响,就好像这里的供电很不稳定一般。

每一次来到这里,苏雨晴都是站在这座公园的门口,回头望去,永远都是漆黑的一片,只有公园才可以进入。

就像是游戏中的一个场景,除了制作出来可供游戏的地方之外,其他都是被空气墙所拦住的,哪怕可以看到其他的景物,也因为空气墙的阻隔而无法过去。

公园有多大,苏雨晴并不知道,因为她只能走进那扇门里,并不能绕着公园走一圈,测一测它的范围。

一踏入公园,就能听见那生锈了的秋千上下晃动所发出的声音,有些像从远方传来的,又有些像从苏雨晴心底里发出的。

虽然感觉有些毛骨悚然,但是苏雨晴知道,那个地方不是很恐怖,而她也去过两次了,没有必要每一次来到这里,都去那个小男孩儿那里。

公园很大,苏雨晴或许可以去其他地方逛逛,说不定能找到她想要找的东西或者是她想要找的答案呢?

于是,苏雨晴就干脆朝着反方向走去,这边的健身器材不多,绿化也稍微少一些,能看到一块一块被分隔出来的空地,以及一座一座小小的凉亭。

这边应该属于公园的小广场一类的地方吧。

苏雨晴想要看一看这个公园有多大,于是就笔直地朝一个方向走,四周的景物变换的幅度不大,而且很快,苏雨晴就走到了尽头。

她看到了公园的围墙,围墙不算很高,但是从围墙向外面看,依然看不见什么东西,所有的,只是漆黑的一片。

围墙并不算太高,但也不算太矮,想要翻过去,光靠蛮力可不行,还得需要一些技巧。

有些时候,人不能总是被思维定势了,说不定翻过这面围墙,就能离开这个梦境的世界,或者找到自己想要找的答案呢?

苏雨晴有些动心了,她四下看了看,终于找到墙上一块有凹痕的地方,然后借着一棵栽种在墙壁旁的粗壮大树爬了上去,一只手抓住墙顶,一只脚踏在墙的凹痕处,然后再让自己坐在墙顶上,就像是坐在一张椅子上,然后换个方向坐一样,苏雨晴就小心翼翼地调整了方向,面向着围墙外面,深吸了一口气,轻轻地跃了下去。

“嗒。”这是苏雨晴落在地上的声音,这片墙后的空间虽然黑暗,但最起码是有实地的……

苏雨晴回过头,还能看到公园里昏暗的灯光,隐约还能听见秋千晃荡着的声音,她又朝前走了两步,再回过头,发现灯光不见了,身后变成了漆黑的一片,就连晃荡秋千的声音也已经没有了。

而当苏雨晴重新看向身前的时候……

她看到的是公园那生锈斑驳的大铁门,此刻正虚掩着,只留下一个正好可供一人通过的缝隙。

“!?”苏雨晴有些惊讶地睁大了眼睛,转过身,摸了摸身后的一片黑暗,那里像是被一堵墙挡住了,但是苏雨晴无论怎么跳,都摸不到“墙”的顶端。

难道即使从公园里翻出来,也无法离开公园的范围吗?

“……算了……还是去找他吧。”苏雨晴轻轻地摇了摇头,放弃了自己继续探索公园的想法。

因为这个梦境总是会让她睡得很沉,万一闹钟响了她还没听见,那上班可就要迟到了呢。

反正每次见到那个小男孩儿,苏雨晴就总能离开这个诡异的梦境的……

“吱呀——吱呀——”苏雨晴穿过丛林走到了一小片空旷的地方,小男孩儿正低着头晃动着秋千,似乎是在思考着什么。

“……”苏雨晴沉默地站在这个小男孩儿面前,她有时候真的想弄明白,这个小男孩儿到底是谁,但是她无论怎么想,都回想不起来——哪怕她觉得再熟悉也不行。

而当离开了梦境,或者说,如果当她不看着小男孩儿的脸的话,她就无法连小男孩儿到底是长什么样子都记不起来……

“小姐姐……你来啦……”小男孩儿缓缓地抬起头,有些虚弱地说道,他的脸有些苍白,额头上还挂着几颗豆大的汗珠。

一阵阴风拂过,让苏雨晴感到一阵莫名的发自内心的恐惧,而小男孩儿额头上的汗珠也被这阵风给带走了。

“呼……”小男孩儿轻轻地晃荡着秋千,看起来比刚才要好一些了,他看着苏雨晴,问道,“小姐姐……你……还在找丢掉的东西吗……”

“……我……不知道我……丢了什么……”苏雨晴有些头疼地说道,而每一次她想这件事的时候,就会有一抹灵光划过,好像那就是答案,而苏雨晴却偏偏怎么也抓不住……

“唔……小姐姐……努力些……你一定能找到的……”小男孩儿说着,摸了摸自己的口袋,苏雨晴以为他会像上次那样掏出一块石头来,但没想到这一次他拿出来的却是一包五颜六色的糖果。

“……这是什么?”

“糖哦……很好吃的糖……小姐姐要吗……”

苏雨晴赶紧摇了摇头,天知道这种五颜六色的东西到底是糖还是毒药呢,哪怕是在梦境中,她也表现得十分小心谨慎。

小男孩儿见苏雨晴不要,也没有说什么,只是从小袋子里掏出一颗红色的小糖果,然后塞进了嘴里,鼓着腮帮,好像很费劲地咀嚼了一会儿,才艰难地将它咽下。

“咿唔!”小男孩儿吃下糖果后,一脸痛苦地低下头了,整个人都蜷缩在了秋千上,而秋千也因为没有了他用力,晃荡的速度变得慢了起来。

小男孩儿看起来很难过、很痛苦,但他还是努力咬着牙,不发出声音,很快,他的衣服和头发就被汗水浸湿了。

片刻后,小男孩儿的痛苦好像减轻了一些,他抬起头来,将脸上的汗水抹去,有些虚脱地倚靠在秋千的铁链上,再次轻轻地晃动了起来。

“……这是……毒药?”苏雨晴问。

“……不是……是糖果……”小男孩儿摇了摇头。

“那你为什么……这么痛苦……”

“只有……吃了它……我……才会觉得……不……痛苦……”小男孩儿断断续续地说道,他看起来真的十分的虚弱,就好像随时都会晕过去一样。

“……?”苏雨晴有些不理解小男孩儿的意思。

“心……心里……疼……吃了它……会……好受……心……不疼……”

也就是说,这是一种能够解决心中哀伤,但又会让身体十分痛苦的糖果吗?

苏雨晴紧盯着小男孩儿,看着他此刻虚弱无力的样子,那种熟悉的感觉也愈发地强烈,脑海中有什么东西闪过,却像一条灵活的鱼一般无法抓住,这种明明感觉答案就在眼前,却抓不住答案的感觉,实在是让人感觉十分的憋闷。

“呲啦——呲啦——”四周的路灯再一次集体开始闪烁了起来,变得明灭不定,整个公园里也是忽明忽暗的……

苏雨晴知道,这就代表着,梦将要结束了,虽然她不知道所见到的这些到底代表了什么,但总之,梦每次都会在自己见到小男孩儿一段时间后结束,这一次也不例外。

“再……再见……小……姐姐……”

苏雨晴摸了摸自己头顶上那即使是在梦中也还戴着的帽子,知道这是小男孩儿把自己当作姐姐的主要原因。

她看着视线里有些朦胧的小男孩儿,突然抓住了那一抹灵光,想要说些什么,但是,梦,结束了。

而苏雨晴刚才刚刚抓住的那抹灵光,也从她的脑海里消失了。

想起来的东西又忘记了,这可比怎么都想不起来的感觉难受多了呢。

……
