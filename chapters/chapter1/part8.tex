\subsection{属于自己的小窝}

“怎么样,小伙子,这间还满意么?”老虎单手搭在苏雨晴的肩上,很是豪迈地问道。

“嗯……!”苏雨晴用力地点了点头。

其实在看到这间房子的第一眼她就有些走不动路了,虽然它真的很小,但对于苏雨晴而言,却意义非凡。

因为这将会是属于自己的小窝,她会有一个真正的私人空间,比如那个柜子,到时候可以摆满自己的女装,自己可以关上门尽情地打扮自己而不用担心被父母发现……

这是一个孩子对自由的向往,对没有约束的生活的美好展望。

哪怕未来并不一定真的会这么好,但这并不妨碍苏雨晴幻想着那份美好。

很多人在小的时候可能都会去寻找过一个秘密基地,或许只是一个大纸箱、或许只是一个无人的小巷……

其实这个时候的孩子,就已经希望能有一份自己的私人空间,就希望自己能独立了。

希望自己能快快长大,其实就是为了能脱离父母的束缚嘛……

哪怕在长大之后,才会怀念那份依偎在父母丰满的羽翼之下的美好。

“满意的话就签下合同吧,一次性最少交三个月,这间房子本来是三百块一个月的,不过既然是老板娘推荐过来的,老娘多少也要卖个人情不是,收你两百块一个月就行了。”

“老虎,上次你租给别人,不是才一百五吗?”

“那个人不一样啊,他的房间里没有卫生间。”

“这孩子多少也是我推荐过来的,而且她还要在我这里打工,年纪轻轻就独自一人生活,经济上肯定很困难,一百五么算了,反正你这房子空着也是空着。”老板娘帮苏雨晴说情道。

苏雨晴站在一旁安静地看着,如果是她自己租的话,肯定会答应两百块钱一个月的房租吧,毕竟苏雨晴不是一个擅长讨价还价的人呢。

“算了算了,那就一百五吧。”老虎摇了摇头,说道,“水电费自理,每个月会找你交一次的。”

“嗯……可是……那个……我……我的钱不够交三个月的……”苏雨晴怯怯地抬起头,看着这个比她稍高一些的“老虎”,小声地说道。

“老板娘,你带来的娃儿,够穷的啊?”

“小晴,你身上的钱够交几个月的呀?”

“一、一个月……”

“那我帮你垫上三百块吧,不过你一个月的工资就没有了,可以吗。”

“谢谢!”

苏雨晴对老板娘万分地感激,现在她觉得,这个和蔼可亲的老板娘,甚至比自己的父母还要亲切一些……

在老板娘的帮助下,苏雨晴顺利地填完了合同缴纳了三个月的租金,现在苏雨晴的口袋里就只剩下一百多块钱了,而且还要考虑到下个月的水电费,所以一定要省着点用才行。

幸好老板娘那里是包吃的,这样一来,生活下去就不成问题了。

生活也终于是稳定了下来,虽然会很艰苦,但是这样自食其力的艰苦,却让苏雨晴感到很幸福。

“好了,那小晴你就先熟悉一下你的房间吧,明天早上六点来店里工作吧。”

“嗯!”

老虎和老板娘都离开了,只有苏雨晴一人待在这小小的房间里。

苏雨晴将房门关上,然后直接整个人躺倒在了什么都没有铺的木板床上。

床很小,即使只是一个人躺在床上,稍稍翻滚几下都有可能摔到床底下去呢。

但苏雨晴却觉得很开心,她在床上向左滚一圈,又向右滚一圈,就这样来回地滚,感觉此刻的自己就是全世界最幸福的人了。

这是只属于苏雨晴一个人的小窝,她可以在里面想做什么就做什么……

七八平米的小房间虽说有些拥挤,但苏雨晴却觉得这份拥挤恰到好处,如果房间很大,一个人住的话反倒会觉得空荡荡的,这样的拥挤,也会让人不会觉得太过寂寞吧。

苏雨晴在床上滚够了之后,又坐在了书桌前,像是要感受一下这种感觉。

夕阳将最后的一抹霞光透过那扇窗户洒在了跛脚的书桌上,苏雨晴眯着眼睛,感受着这一天中最后的温暖。

接下来,就是有些冰冷而漫长的夜了呢。

一直到夕阳完全落下,苏雨晴心中刚才的那些因为开始了独立生活而感到兴奋的心情也平复了许多,接下来就是该布置这属于自己的小窝的时间了。

苏雨晴把行李箱打开,里面塞满了自己带出来的东西,并不只是衣物,还有一些她特别喜欢的东西,比如那只小猫的毛绒娃娃。

苏雨晴把行李箱的东西一样一样地拿出来,整理好、分好类,放在空荡荡的木板床上。

平角裤就和平角裤放在一起,衬衫就和衬衫放在一起,长裤就和长裤放在一起……

苏雨晴耐心地折叠着这些衣物,将它们全都叠得整整齐齐之后,才放进了那个干干净净的衣柜里。

看得出来,老虎可能是有些轻微的洁癖,因为这间房子虽然没有租出去,但是她可能也是偶尔会来打扫一次的,地上、桌上、柜子上虽然有灰尘,但并不多,用手拍拍就能掸去了。

除了衣物外,还有一张薄薄的毯子,苏雨晴将它铺在了床上,棉被因为太大太重,所以她就没有带,看来到时候还得去买床棉被回来。

“不知道这附近有没有便宜些的棉被卖呢……”苏雨晴一边铺着自己的床,一边自言自语地说着。

春天的晚上还是很冷的呢,苏雨晴可不想像昨天晚上那样被冻一个晚上……

床单的颜色是素色的,上面有着淡粉色的花纹,原本简陋的房间在铺上床单之后,一下子就温馨了许多。

苏雨晴把起司猫的毛绒娃娃摆在了床头,然后将其他的一些东西也一股脑的拿了出来。

比如她很喜欢的一个木制笔筒,被她摆在了桌上,还有几个可爱的卡通人物手办,虽然做工很粗糙,但都是苏雨晴非常喜欢的东西。

除此之外,还有一本厚厚的本子,那是苏雨晴的日记本,已经密密麻麻地写了好多页。

每当内心空虚的时候,看一看自己写的日记,就感觉好像是隔着时空和以前的自己在对话呢,多少也能减轻些心中的寂寥吧。

对于苏雨晴来说,日记也是必备的东西。

行李箱装得满满当当的东西,真的全部拿出来之后,其实也并不多了。

除了那些东西之外,就是一些零碎的物件儿。

比如一块很可爱的毛巾、一本烹饪的食谱、一部苏雨晴翻阅了许多遍的小说、一部鲁迅的散文集册……

东西不多,但分散着装点在房间的各处,就有一种十分温馨的感觉了。

空空如也的行李箱被苏雨晴放到了床底下,书包也被她放进了衣柜里。

将这一切全都完成之后,苏雨晴伸了一个小小的懒腰,心满意足地看着自己的房间傻笑了起来。

“真好……我自己的房间呐……”

开始独立的成就感,将苏雨晴心中的负面情绪也冲散了不少。

“生活会变得越来越美好的……嗯……一定会的!”苏雨晴捏了捏拳头,脸上洋溢的笑容让此时的她看起来是那样的阳光并且积极向上。

苏雨晴把口袋里的钱都掏了出来,摆在桌上仔细地数了起来,一共还有一百四十九块钱,她将其中二十块钱叠好放进一个小信封里,这是用来交下个月水电费的钱,要先放好。

这样子就还剩下一百二十九块钱了。

“唔……这一百块就去买一点生活用品,二十九块的零钱就放着零用吧……”苏雨晴细心地分配着这些钱,只有有计划地花钱,才不会让自己陷入没有钱用的窘境呢。

哪怕面馆里提供三餐,但也不能把身上的钱全都花光嘛。

万一有个什么事情呢?比如感冒啦,发烧啦,肚子疼啦……总要去买药什么的吧。

将目光放得远,计划好未来的事情,也代表着苏雨晴在慢慢地成长,这是成熟的开始……

苏雨晴的小窝的地板是水泥地,哪怕打扫干净了也不能赤脚踩在上面,这样子这些空地也会有所浪费,虽然空出来的面积并不多……

但是每一分空间都得利用到位才是嘛。

“不知道有没有卖那种泡沫拼图的……垫在地上的话,应该很不错吧……”苏雨晴小声地自言自语着,拿着自己那最后一张一百元的大钞,拿上房东刚刚给她的钥匙,走出了家。

嗯,从现在开始,这里就是她的家了,只属于她一个人的家。

晚上的时间,街道上人明显要比白天要多许多,也热闹许多,大概是因为大多数人都下班了的缘故吧。

街上各种各样的店还是很齐全的,苏雨晴很轻易地就找到了一家卖被子的小店。

小城市的物价很低,一床一米八的普通厚薄,一般用来春天盖或者秋天盖的棉被,也只要七十块钱一条。

这是一家富有小城市气息的床单店,东西虽然便宜,但是被子的款式和花样实在是太过老气了,苏雨晴想要找一条稍微可爱点的被子,愣是半天都没找到一条。

“小姑娘,看中了哪条?”老板刚做完一单生意,满脸笑容地凑到了苏雨晴的身旁,问道。

“唔……那个……有没有可爱一点的被子……”

……
