\subsection{莫空}

这是一如既往平静的一天,今天没有下雨,算是半个阴天吧。

之所以说是半个,因为太阳偶尔还是会冒出来露个脸,然后又钻到云层里去的。

这样的天气是最适合到外面玩的,而且现在是春天,更是适合游山玩水什么的了。

小城市是一座靠海的城市,而苏雨晴却还没有去过这里的海边呢。

苏雨晴不喜欢那种到处都是人的沙滩,或许小城市这里能让她领略一下无人的沙滩风光吧?

要不等下个星期一,就和月橙去海边玩好了?

“我的牛肉面好了没啊?”一位坐在桌前等了好一会儿的客人有些不耐烦地喊道。

“啊啊,来了!”苏雨晴赶紧从发呆的状态中回过神来,端起厨房小桌上的一碗牛肉面,小心翼翼地捧到了那位顾客所坐着的桌子上。

天气好,生意也会好很多,今天的客人就比昨天多得多了,甚至苏雨晴到这里以后,除了吃了早餐外,就忙得没有停下来的时间,从九点就有不少客人了,现在是十一点,客人更加多了,有些人见桌子满了坐不下,干脆直接打包带走吃……

“小晴!去超市里买一块一包的雪菜!买十包!雪菜不够了!”张阿姨在厨房里大喊道。

“来了……”苏雨晴赶紧跑回到厨房,接过张阿姨给的十块钱纸钞,就火急火燎地朝不远处的一家小超市跑去了。

苏雨晴不太喜欢生意特别好的时候,因为那样会让她感觉有些手忙脚乱的,总是怕把事情做砸了,而越是这样想,也就越是紧张,反而速度不够快,也不够好了。

当然,像昨天那样冷冷清清的也无聊,那样子就感觉时间过得太慢了。

苏雨晴喜欢的是那种普通的生意,客人不少,但也不多,可以不急不缓的,不用那么匆忙。

那家小超市和面馆在同一条街上,距离并不算太远,大概也就两三百米的样子吧,但因为跑得太快,再加上苏雨晴的体力本身就不好,在跑到超市门口的时候,她就已经气都快喘不上来了。

只是时间紧张,苏雨晴来不及休息,就一边大口地喘着气,一边跑进了超市里,迅速地找到了卖雪菜的货架,找了一块一包的那种,拿了十包,再又冲到了收银台前。

付完钱后,又是一路狂奔,跑回到了面馆。

苏雨晴将雪菜放到了厨房里的小桌上,然后就扶着墙大口地喘着气,这时候的她感觉连站都快站不稳了呢……

“呼……呼……好累……”苏雨晴擦了擦额头上的汗,尽量让自己多休息一会儿,因为在这么忙的时候,休息的时间可是很宝贵的。

来回大概五六百米的距离不算远也不算近,虽说苏雨晴体质本身就差,但是她已经感觉到身体比以前还要虚弱,而且是短短的那么几天就虚弱了这么多。

以前就算跑一千米都不至于累得喘不上气来,而现在才五六百米,就感觉胸口上好像压了一座山,连呼吸都十分的困难,眼前的视线都是模糊的,大脑也是有些昏昏沉沉的,就好像一下子饿了三天一样。

苏雨晴缓缓地蹲了下来,靠着墙根大口地喘着气,厨房里的油烟让她觉得很难受,但是没办法,此时的她实在是浑身乏力,连挪动身子走到外面的力气都没有了。

“小晴,小桌上那碗雪菜肉丝面端上去,是六号桌最靠墙的那位客人。”正忙碌着的张阿姨没有看到苏雨晴此刻的样子,只是背对着她吩咐道。

像这么忙的时候,都是老板和老板娘一起上阵的,厨房里有两个炉灶,同时烧面的速度肯定比一个人做要快得多,就算是这样,也只是勉强跟得上点菜的速度而已。

今天的客人来得多,吃面的速度也快,几乎是一批接一批,都没个停顿的,忙得连桌上的碗筷都快来不及收了。

“嗯……”苏雨晴深吸了一口气,捂着因为剧烈运动而疼得难受的肚子缓缓地站了起来。

这就是工作,只要还在工作,哪怕再难受也得做,苏雨晴不喜欢把自己表现得太过柔弱,那样人家会觉得她很没用,说不定哪天就把她解雇了。

所以很多事情,哪怕很累,她都咬着牙坚持下来。

这一次也不例外。

刚刚短暂的休息让苏雨晴的呼吸稍微顺畅了一些,压着胸口的东西仿佛从一座山变成了一块大石头,虽说还是难受,但总归是好得多了。

苏雨晴双手有些颤抖地捧起那碗面,此刻她自己都有点走不稳,更何况是捧着一碗面,走得摇摇晃晃的,而且她越是想保持平衡,晃得就越厉害,甚至有几滴滚烫的面汤落到了她的手上,那种烧伤般的疼痛让苏雨晴恨不得直接撒手把这碗面给扔了,但她还是得咬着牙忍住……

苏雨晴总算是慢慢地走到了那位客人面前,此刻她的全部注意力都集中在这碗面上,见自己总算到了,这才小小地松了口气。

“您的面……”苏雨晴刚刚想放松一下,却因为这份疏忽而没有捧稳面汤,左手有些倾斜,眼看面就要全都洒出来了,而且还很有可能洒到别人的身上……

“小心点。”就在苏雨晴的心都快跳出来的时候,一只宽厚而有些粗糙的大手轻轻地托住了苏雨晴的左手,将那倾斜的角度修正了,然后另一只手伸了出来,接下了这碗雪菜肉丝面,稳稳当当地放在了桌上。

苏雨晴抬起头,看到的是一个有些深邃而又有些沧桑的双眸,他是一个乍一看好像有些特别,但仔细看好像又十分普通的男人。

没错,就是那个给苏雨晴讲过故事,或许还有可能是在火车里叫醒她的男人,一个不修边幅的年轻男人。

“谢、谢谢!”苏雨晴的小脸涨得通红,不仅是因为尴尬,还有几分害羞,因为刚才年轻男人的手托住面碗的时候,可是直接整个抓住苏雨晴的右手的呢……

苏雨晴觉得心跳得有些厉害,虽然她之前的心跳就没有从剧烈运动中恢复,但总感觉这个时候的心跳似乎和之前有些不太一样……

“没事,你去忙吧。”

苏雨晴低着头,犹豫了一下,小心翼翼地抬起头来,最后还是支支吾吾地主动问道:“那……那个……你……你叫什么……咳嗯……那个……”

“哦,我的名字啊,上次忘了和你说了,我叫莫空,你也可以叫我莫成。”

“两……两个名字?”苏雨晴有些疑惑。

“莫空是我身份证上的名字,而莫成,是我自己给自己取的名字,或许,你可以把它当作我的笔名吧。”

“莫……莫空……?”苏雨晴刚说完他的名字,就再次害羞地把头垂了下去。

“嗯,莫名的莫,天空的空。”莫空从一旁的筷子篓里拿出一双一次性筷子,一边拆,一边问道,“不介意的话,可以也告诉我你的名字吗?”

“我、啊……那个……我叫……苏……苏苏苏……雨晴……”苏雨晴有些慌乱地说道。

“苏州的苏,下雨的雨,晴朗的晴?”莫空问。

“嗯……”

“很好听的名字啊。”莫空笑着说道。

“小晴——这碗炒面拿过去——”厨房里传来了老板娘的声音。

“快去忙吧。”莫空看着苏雨晴,温和地笑道。

“嗯、嗯……”苏雨晴如蒙大赦般地跑回到了厨房里,只是松了一口气之余,好像还有些隐隐的不舍?

难道苏雨晴还想要和莫空再多聊两句吗?

生意很好,面馆很忙,忙得苏雨晴都快停止了思考,只知道机械式地按照老板娘的吩咐做事。

同样的,也暂时地忘记了莫空的事情。

而等到她停下来的时候,又哪有莫空的身影了。

也是,来面馆吃面的食客,吃完了面肯定就走了嘛,更何况之前还是高峰期,如果不吃面还占着位置的话,要是遇到脾气暴躁点的,估计直接打人了吧……

下午两点半,面馆里只剩下一位还在悠闲地吃着面的客人了,忙碌的苏雨晴也总算是稍微松口气了。

接下来就不用那样急匆匆地了,只要慢慢地收拾一下残局就可以了。

因为之前太忙,连碗筷都来不及收,都是直接推到一旁然后给客人上面,除非是堆得很满了,才会赶紧跑来收拾掉。

望着眼前一张张狼藉的餐桌,苏雨晴不禁有一种人去楼空的伤感。

其实应该高兴才对吧……

碗筷收拾好后,要丢到大脸盆里去进行清洗,除此之外还有擦桌子,收拾椅子,补充调味料之类各种各样琐碎的事情要做……

幸好晚上的生意就没有白天那么火爆了,虽然人也不少的,但也是在正常范围之内。

很快,就到了下班的时间。

苏雨晴在面馆门口伸展了一下身体,感觉浑身都酸痛得要命,这要是每天都这么忙、这么累的话,恐怕苏雨晴支撑不了多少天就得因为太过疲劳而生病了吧……

“小晴,累了吧。”

“嗯……好累……”苏雨晴诚实地说道。

老板娘将几盒牛奶放进袋子里递给苏雨晴,道:“这几盒牛奶带回去喝吧,你还在发育的时候呢,营养不能少,回去就赶紧休息吧,不然明天会更累的。”

“嗯……!谢谢老板娘……”

……
