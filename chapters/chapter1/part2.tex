\subsection{小城市}

二零零四年,是网络及计算机高速发展——或者说,高速普及的年代,各种各样的地方都开始用上了计算机,只不过相对来说比较传统的火车站距离它开通网上售票的服务还需要好几年。

而且在这个年代,买火车票也不需要出示身份证,一个人可以一口气买上好多张,黄牛也因此而特别泛滥。

当然了,在杭州这个大城市里,黄牛相对而言还是比较少的。

“到哪里。”售票员冷冰冰地说道,本应该是一句问句,从她的嘴里说出来,就变成了一句陈述句。

“……”苏雨晴沉默着,因为她也不知道自己要去哪里,或许,应该去一个很远很远的地方?

但是那个地方应该是哪里呢?北京?成都?

太远了,或许生活上会有很多不习惯吧,而且那些大城市的物价肯定很高,苏雨晴带的钱总共也就一千五,恐怕在那里连一个月都用不了吧。

那么,在浙江省的范围内,又距离杭州比较远的,物价比较低的城市是哪个呢?

毫无疑问,就是小城市了。

小城市的名字就叫做“小城市”,如它的名字一样,它不大,物价也是有名的低,生活节奏很慢,只是现代化程度完全没法和杭州相比就是了。

“请问去哪里?”售票员有些不耐烦了,她再问了一次,恐怕第三次她会直接将苏雨晴让开别挡路吧。

“去小城市……”苏雨晴有些怯怯地说道,这还是她第一次自己买火车票,看到售票员那样冷淡的眼神后,顿时不知道该用怎么样的语气去应对了。

无论怎么说,苏雨晴终究只是一个孩子而已。

她很胆小,但往往胆小的人,才会做出那些胆大的人都不敢做的事吧?

比如,离家出走,在她这个年龄段,敢这么做的很少,而且苏雨晴还不是一般的离家出走,她是真的想要离开自己的家,去外面的世界闯荡、生活。

她知道未来的生活会很困难,但是,她恐怕不会想到,未来的生活,比她想象中的还要困难十倍、百倍。

车票还有些热乎乎的,可能是刚打印出来的缘故吧。

苏雨晴紧紧地捏着车票,随着人群向里面走。

杭州的城站火车站实在是太大了,最起码对于她这个年龄段的孩子而言,是属于那种可以迷路的程度了,苏雨晴没有去问路,不是因为自尊,而是因为胆怯。

她其实是一个不擅长和人交流的孩子呢。

也不知道是自闭症让她想要变成一个女孩子,还是因为想要变成女孩子,才患上了自闭症呢。

“从杭州开往小城市的 T136 号列车即将开动了,请还未检票的乘客去 4 号窗口检票。”

广播的女声响了起来,苏雨晴攥着车票的手有些冒汗了,因为她到现在还没有找到 T136 号列车在哪里上车,只是听到了广播的声音而已。

苏雨晴四下张望着,额头上布上了少许细密的汗珠。

“小姑娘,找不到上车的口子了吗?”一个温和而有些关切的声音从一旁传来。

苏雨晴“刷”地一下扭过了头去,有些紧张地看着刚才说话的那个人——一位正在打扫卫生的中年妇女。

“小姑娘,有什么困难,我可以帮助你吗?”

“我……我……”苏雨晴的小脸有些微红,她鼓足勇气说道,“那个……我要坐这班列车……可是、可是找不到……入口……”

扫地大妈看了看苏雨晴递来给她看的票,然后抬起头来,笑道:“就在前面,往右边进去就是了,哪里有数字牌的,叫到是哪个窗口检票,你就去哪个窗口检票,就可以了。”

“谢谢!”苏雨晴感激地朝大妈道了声谢,拖着自己的行李箱飞快地跑到了检票口。

只听到检票员拿着检票器“咔嚓”一声在票上订了一个洞,然后朝苏雨晴摆了摆手,道:“快点,车要开了。”

“谢谢!”

和其他人进行了简短的交流之后,苏雨晴的心情已经平复了许多,那些害怕的情绪也消退了一些。

似乎,一个人在外面生活,也不会很难吧,因为,这个世界上还是有很多好心人的呢。

但是苏雨晴不知道,这个世界上,有多少好心人,就有多少不怀好意的人……

生活,永远都不会那么简单,在这条旅程上,她还要走很远、很远……

“咕咚、咕咚、咕咚——”有些老式的火车缓缓地开动了起来,这是特快列车,在高铁和动车还没普及的年代,就代表着火车最快的速度了。

苏雨晴顺着车票找到位置坐了下来,她担心自己的行李被拿走,所以干脆把它放在了自己的脚边,这样最起码能随时看着它了。

今天并不是什么节假日,也不是什么特别重要的日子,坐火车的人不算很多,或者说,坐从杭州到小城市的火车的人不算很多。

这是直达列车,起始站是杭州,终点站是小城市。

毕竟现在不是返乡时期,更多的人应该是从乡下到大城市里来,而不是回乡下去。

在很多人眼里,小城市这样落后的城市,和乡下的区别也不算很大吧。

火车开的速度很快,但是苏雨晴却不觉得四周的景物倒退的速度有多快,她依然能清晰地看到树上的鸟巢,田间的蟋蟀……

独自一人的旅行,带给了苏雨晴些许的恐慌和紧张,但在这之中还有着些许的兴奋。

是因为自己终于自由了而感到兴奋,还是因为这是自己第一次独自旅行而感到兴奋?

或许,两者皆有吧。

苏雨晴幻想着未来的美好生活,到达小城市后,她可以找一个工作,然后租一个属于自己的小房子,可以尽情地穿那些漂亮的衣服,尽情地吃那些能让她变得更“漂亮”的药,而不用担心身体的变化会让父母发现。

事实上苏雨晴此时的身体状况和药虽然有点关系,但关系并不算特别大,她本身就是发育比较晚的孩子。

那些药她也只是每个星期,甚至半个月才吃一次而已,主要就是担心身体的变化让父母察觉。

只是没想到,身体还没发生太大的变化,就已经被父母发现了自己的秘密呢。

药物的作用并不明显,顶多只是让她原本就迟的发育更加推迟了而已。

苏雨晴本就属于那种长得比较漂亮的男孩子嘛,她的父亲在年轻的时候,也是一个美男子呢。

一想到自己的父母,苏雨晴因为独立而带来的兴奋感一下子就褪去了大半,她托着下巴,脑海中出现了父母的图像。

她恨自己的父母不能接受自己的孩子;她恨自己的父母不能认可她的做法;她恨自己的父母强硬地拉着自己去剃了一个只属于男孩子的平头短发……

但是她也有些后悔,后悔自己或许不应该那么冲动,等到父母回家之后,她们会怎么做?寻找自己?可是,真的找得到吗?

苏雨晴的情绪是复杂的,她既希望自己的父母能找到自己,又希望她们永远找不到自己,一直到把自己忘掉。

她,毕竟只是个孩子而已。

“给你一个晚上好好想想,到底是做我的儿子,还是滚出这个家!”

父亲的话在苏雨晴的脑海中回响,她再次变得坚定起来,默默地捏紧了拳头,小声地自言自语道:“我绝对不会再回去了,绝对……”

窗外的风景似乎也变得不再那么吸引人了,千篇一律的山水农田让苏雨晴觉得有些困倦。

一个人出行,最担心的就是坐车睡过头,但是终点站是小城市,应该没有问题吧?

苏雨晴的小脑袋一点一点的,像小鸡啄米一样,最后终于抵挡不住那精神上的疲倦,脑袋向后一仰,倚靠在座椅上睡着了。

“咕咚、咕咚、咕咚——”苏雨晴的梦里什么也没有,只有这火车不断行进时所发出的声音,像是一首美妙的乐曲,让她睡得愈发的安稳,伴随着火车的摇晃,让她感觉像是回到了小时候,回到了母亲的怀抱里,听着母亲轻声哼唱的摇篮曲一样。

坐在苏雨晴旁边的年轻男人看着苏雨晴此时恬静的睡颜,不由得有些痴了,他伸出手,想要摘下苏雨晴的帽子,看看她那柔顺的长发,但最终还是没敢乱动,悬在半空中后,又收了回来。

“真可爱啊……为什么,没有大人一起陪同她出来呢?”年轻人歪着头想了一会儿,脑海中似乎划过了许多可能,但他却没有下结论。

“哈……还是改不掉看到人就想她的经历这种毛病啊……”年轻人轻笑了两声,将头扭向了别处,窗外,一排排白杨树向后倒去,天空依然阴沉沉的,恐怕今天一整天,太阳都不打算出来了吧。

“嗯……阴天也好,阴天嘛……别有一番风味……”

年轻人又将头扭向了苏雨晴,注视了她好一会儿,像是要把她的容貌完完全全地刻在自己的脑海里一样。

“真是可爱的女孩儿……那些小说的主角们所想要守护着的女孩儿,应该就是像她这样的吧?”

“咕咚、咕咚、咕咚——”列车开着,车厢里很温暖,温暖得让苏雨晴想要就这么一直睡下去,永远也不要走到那阴冷的外面世界里去。

“喂,醒醒,终点站到了,该下车了。”一个年轻男人的声音,破坏了苏雨晴那虚无而安逸的梦境。

……
