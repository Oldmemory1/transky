\subsection{第一次上班(下)}

“小晴,来啦。”正在择菜的老板娘抬头看着苏雨晴,笑着问道。

“嗯……老板娘……”

“以后叫我张阿姨就可以了。”

“那……老板呢?”

“叫李叔叔。”低着头算着今天早上买菜的菜钱的李老板也抬起头来,朝着苏雨晴温和地一笑。

看得出来,这二人都是比较温和善良的人。

他们的笑容也让苏雨晴觉得温暖了许多,本来因为担心第一天工作做不好而有些僵硬的身子也恢复了许多。

“小晴,先吃早餐吧。”

“诶?不先干活吗?”

“待会儿可就要忙起来了,不吃饱可没有力气干活哦。”

“嗯……好的……”

“小晴想吃什么面?叔叔给你烧。”李老板用铁勺子敲了敲锅子,大笑道。

“姆……随便什么都可以……”

“那就给你来碗拌面加个荷包蛋怎么样?”

“好的……”

拌面很快就好了,李老板还特意为苏雨晴把面拌好端到了她坐着的桌前。

面用的是那种粗的圆面,红褐色的酱油均匀地附着在每一根面条上,在面条上盖着一个鲜嫩的荷包蛋,蛋黄没有完全凝固,而是处于半流质的状态,在最上面还撒了些许的葱花,香味四溢。

“谢谢……”苏雨晴发自内心地感激道,感觉老板和老板娘就像是自己的再生父母一样。

“多吃点,不够再给你烧,小伙子,吃得壮一点,看你这柔柔弱弱的,以后可找不到老婆啊。”李老板开玩笑地说道。

事实上,苏雨晴还真没有想要找老婆的念头呢……在心底里,她可是把自己当作女孩子看待的呢。

“真的……真的很感谢你们,又给我工作,又给我吃的……比我的父母还好……”

一直在择菜的张阿姨将处理好的菜放进了篮子里,双眼中流露出些许思考的神情。

“我们只是给了你工作和食物,你就这么感激我们了,那就应该更感激你的父母呀,无论如何,最起码是她们将你带到了这个世界上来呢。”张阿姨觉得苏雨晴的心理状态似乎有些问题,出于女人的天性,她开始耐心地劝导了起来。

“嗯……”苏雨晴没搭话,其实她巴不得自己没有出生在这个世界上,那也就不会有如此纠结的自己了,为什么父母要给自己男孩儿的身体?

只是这些事都不能和其他人说,苏雨晴只能将它们藏在心底,表面上装作什么事都没有的样子。

“小晴,我看你似乎有什么心事,如果不介意的话,就和阿姨说说吧?”

“没什么……”苏雨晴摇了摇头,即使张阿姨人很好,她也不打算将这些事情告诉她,因为她想要做的事情对于普通人来说,实在是太过惊世骇俗,实在是太过骇人听闻了……

一个男孩子,想要成为一个女孩子?恐怕在很多人眼里,这都是大逆不道的事情,这都是精神病人的作风吧。

不过话说回来,哪一个想做女孩子的男孩子没有点心理疾病呢?

经过这件小事一打断,本应该十分美味的拌面在苏雨晴嘴中也失去了味道,就像是在嚼着蜡烛一样,气氛一下子就变得沉默了起来。

不过沉默的时间并不久,因为很快就有客人来了。

这家面馆兼做早餐生意,虽说没有普通的早餐店那么丰富,早餐的种类也就只有拌面、馄饨、水饺那么几种,但来吃的人却还真不少,也难怪这家店的老板要招一个帮工了。

苏雨晴的工作很简单,就是负责记下客人点的菜名,然后告诉厨房里的老板和老板娘,再把菜端上去,以及把收到的钱交给老板娘之类的事情。

听起来挺简单的,但实际上却是很忙碌的。

“哟,老板,你们招了个漂亮的小妹当帮工吗?啧啧,看这脸,还没成年吧?”有经常来面馆点餐的食客大笑着调侃道。

李老板的声音从厨房外的小窗口中传出来,他也大笑着说道:“是啊,看看,长得漂亮吧?要不给你家儿子当媳妇?”

“嘿,那感情好啊。”中年男人看向苏雨晴,笑着问道,“小姑娘,要不要给我家儿子当媳妇啊?当了媳妇你就不用工作了,我们家养你。”

“哈哈!老王最近做生意又发财了哈?”

“老王你凑个什么热闹,你家儿子才几岁啊,比这姑娘还小吧?”

“嘿,会不会老王其实想把这小姑娘娶回去当小老婆啊?”

常来的食客们纷纷调侃道。

苏雨晴的小脸通红,她有些结结巴巴地对那个中年男人问道:“请、请问……要点什么……”

“来碗大馄饨,多加点葱哈。”

“李叔叔,大馄饨,多加点葱。”

“啥?”面馆里太过嘈杂,正在厨房里忙碌的李老板听不清苏雨晴的声音,当然,这和苏雨晴的声音实在是太小了也有关系。

“大馄饨……多加点葱……”苏雨晴把小手摁在喉咙上,努力地大声说道。

或许是因为她很久都没有大声说话过了,所以即使努力大声喊,发出的声音也并不大,好在一旁的老板娘张阿姨听见了,转述给了李老板,苏雨晴这才松了口气,那样的大声喊,她甚至会觉得喉咙有些难受呢。

“小晴,这份拌面端上去。”

“哦。”

苏雨晴端起拌面,小心翼翼地走到了一位坐着等待的食客身旁,虽说就那么点时间,不至于忘了是谁点的拌面,但为了保险起见,苏雨晴还是问了一句“请问是您要的拌面吗”,在得到肯定的回答之后,才将拌面放到他面前的桌子上。

一整个早上都相当的忙碌,苏雨晴已经不知道自己端上去几碗拌面、几碗馄饨、几碗水饺了……

她甚至都觉得精神有些麻木了,甚至有时候没有人点餐,她都会下意识地走到厨房里,想要说是谁点了什么,然后刚准备张口的时候才想起刚才没有人点东西……

早上的早餐时间很热闹,但是也很短,八点半之后,来吃早餐的客人就越来越少了,到了九点钟,面馆里就彻底安静了下来,一位食客也没有了。

苏雨晴终于获得了短暂的休息时间,虽然只是短短的几个小时,但是在店里来回走动可是相当累的呢,苏雨晴感觉自己脚底上的水泡更痛了,本来昨天就没有消退……

再这样下去,估计会磨出老茧来吧……

“累了吗?”张阿姨看着有些疲惫地瘫软在椅子上的苏雨晴,笑着问道。

“嗯……脚好痛……”

“习惯就好。”

虽说苏雨晴长得很漂亮,但毕竟在张阿姨的眼里还是一个男孩子,对于老一辈的人来说,男孩子吃点苦,都不算什么,或者说,反而是成长的路上应该受的呢。

“嗯……”苏雨晴轻轻地点了点头,虽然她想成为女孩子,但不代表她希望被别人看不起,其实她也知道这种工作真的算不上有多累,只是她自己的身体实在是太过娇弱了而已。

付出努力,才能获得工资。

“小晴,休息好了就把桌子擦一下吧。”

“好的……”

苏雨晴咬着牙站了起来,拿起抹布开始细心地擦起了桌子。

自己赚钱养活自己,并不是想象中那样简单轻松的事情呢。

苏雨晴还是把一切都看得太过容易了。

好在面馆到中午都不会有太多生意,苏雨晴擦完桌子后总算可以休息一会儿了,碗都是堆在那等到下午再洗的,因为下午有比较长的时间没有什么生意嘛。

做完了事情的苏雨晴就坐在了靠近店门口的位置上,这里偶尔会有清风拂过,但又不会让人感觉太冷。

椅子有点高,苏雨晴坐在上面要踮起脚尖才能碰到地面,她就这样坐在上面,轻轻地摇晃着身子,吹着被温暖的阳光“加热”过的清风,上下摆动着双腿,一副悠闲自在的模样。

在一番忙碌之后,然后坐着休息一会儿,这种偷得浮生半日闲的感觉,实在是太过美妙了呢……

“你好,还有早餐吗?”一个不修边幅的年轻男人走进了面馆,朝店里面喊问道。

“要点什么?”苏雨晴赶紧从椅子上站了起来,问道。

不修边幅的男人疑惑地看了苏雨晴一眼,问道:“你是……老板?”

“……”

“要点什么?”关键时刻,还是张阿姨来解围,因为不善言辞的苏雨晴一时都不知道该怎么回答他的问题。

“拌面加荷包蛋再来碗馄饨。”

“小伙子胃口不错啊?”李老板笑道。

“饿了嘛,胃口自然就好。”不修边幅的年轻男人微微地笑了笑,在苏雨晴边上的桌子前坐了下来。

“你是来这里打工的吗?”他问。

“嗯……”

“……打工可是很辛苦的,要加油啊。”出乎意料的,不修边幅的年轻男人并没有问苏雨晴的年龄,只是轻描淡写地说了一句。

苏雨晴有些疑惑地看着这个在她看来有些奇怪的男人,觉得自己好像在什么地方见过他似的。

嘛,或许又是错觉吧。

想不清楚的东西就不去想,反正和自己没什么关系,苏雨晴帮不修边幅的年轻男人端来拌面和馄饨之后,再次坐到了椅子上,眯着眼睛“享受”了起来。

虽然工作累了点,但如果习惯了的话,其实这样的生活也是挺好的吧?

……
