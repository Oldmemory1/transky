\subsection{做客的猫}

天气的逐渐暖和,对于苏雨晴来说也是一件好事,最起码洗澡不用烧好几壶热水了,因为天气温暖,水温不用像前几天温度低的时候那么烫,只要适中就可以了,甚至,如果是体质好一点的人,洗冷水澡都没有问题呢。

卫生间本身就很小,除去马桶之外,可供站立的地方就更少了。

就算是苏雨晴这么娇小的身材,在里面洗澡都觉得有些施展不开。

“要是有钱就好了……”苏雨晴搬了一张小凳子坐在卫生间狭小的空地上,自言自语地说道,“好想泡澡……好想痛痛快快地洗个澡……”

有钱的话,就可以去浴室洗澡,如果再有钱一点的话,去蒸个桑拿也不成问题。

只是苏雨晴现在最缺的,就是钱呢。

以前手里不缺钱用的时候感觉自己对钱不怎么依赖,而现在钱不够用的时候,却恨不得将一块钱掰成两块钱花呢。

苏雨晴轻轻地搓洗着自己的大腿,或许是雌性激素的作用,又或许是药物的作用,她总觉得大腿的皮肤变得更细腻了,哪怕是贴近了看,也难以看到毛孔了。

摸着自己的大腿,苏雨晴的心中产生了些许异样的感觉……

如果太过用力,会觉得有点疼,但如果比较轻柔的话,就会感觉一阵酥软,就好像别人在给她挠痒痒一样。

按理来说,自己摸自己应该是不会觉得痒的,除非是身体太过敏感了……

“哗啦——”苏雨晴关掉了淋浴喷头,用干毛巾盖在自己的头上擦了擦头发,头发短也有头发短的好处,最起码不用吹风机,一下子就能干了……

虽然只过去一个星期左右的时间,但苏雨晴的头发已经长了些许,虽然还是平头的模样,但比刚剃完头时“丑陋”的样子要好了不少。

而且苏雨晴能明显感觉到,新长出来的这几毫米的头发,明显比之前的头发要细得多,只是因为还太短,表现得还不是很明显,说不定等以后苏雨晴就能有一头柔顺的长发?

这大概也是药物的作用吧。

像苏雨晴这一类的人,肯定会有一个愿望,那就是希望自己能拥有一头像真正的女孩子那样柔顺的长发,而且得是自己长出来的,假发什么的,终究不是自己的东西呢。

“快点长、快点长……”苏雨晴对着镜子里的自己鼓劲道,就像是一位在念咒语的魔法师,好像说了以后,头发就能很快变长了。

头发长得可是很慢的,哪怕是速度比较快的,想要长到齐耳的长度,最起码也要一两个月的时间,慢的话三个月甚至半年也有可能,也难怪当时的苏雨晴会因为自己一头齐耳短发被剃成平头而那么愤恨了。

这就好像你精心制作了大半年的东西,却被人拆散让你重新再做一遍,那种滋味,可真的是会让人有些发狂的呢。

痘痘几乎快要消退完了,只剩下一两颗不太明显的,就算不用帽子遮着都不一定看得出来。

“呼……总算快没有了。”苏雨晴摸了摸自己光滑的额头,只有在有痘痘的地方才有凸起,而过两天这痘痘就会消失了,也让她不禁有些高兴。

相信大多数人如果脸上长了痘痘,都会觉得有那么些不爽的吧,更何况是“爱美”的苏雨晴呢。

洗完澡后就立刻把衣服洗掉,苏雨晴不喜欢把事情拖着,她喜欢今天的事情今天完成,如果不完成的话,甚至连睡觉都会觉得不安稳呢。

“十点了……”苏雨晴看了看闹钟,又看了看铺得整整齐齐的床铺,却觉得今天自己的精神格外的好,似乎并不怎么想睡觉的样子。

就这样躺在床上也睡不着,反而会更加清醒,还是找些事做,等困了再睡吧。

苏雨晴这么想着,就坐到了书桌上,写起了日记,内容无非就是关于那个论坛里的好友竟然和自己在一个城市里的事情。

日记也很快就写完了,可苏雨晴还是不觉得困,于是她就在一个涂鸦本上画起了画来。

苏雨晴小时候学过画画,只是后来因为学业太忙而没有继续了,但是她对于画画还是很感兴趣的,无聊的时候就会在涂鸦本上画上一会儿,说是在画画,其实是在一边画,一边梳理着自己的心情吧。

不要以为有钱人家的小孩日子就一定过得很舒服,苏雨晴的童年其实并不怎么快乐,从幼儿园开始就要面对大量的课外兴趣班,双休日都只有半天的休息时间。

到了小学之后就更是变本加厉了,一年到头,哪怕是暑假寒假,都几乎没有休息的时候。

书法、绘画、钢琴、古筝、奥数、舞蹈、跆拳道……

这些东西把苏雨晴的业余时间占得满满的,根本就没有喘息的时间。

有钱人不缺钱,所以更要尽力地把自己的孩子培养成全方面的人才,只是,世界上哪里有完美的人,一个人能有一两个特长就已经很不错了,而大多数的中国人却想要把自己的孩子培养得什么都会,那不是在说笑话么?

那些所谓的全方面发展培养的孩子,又有几个成功的了?在成年之后,还不是依然泯然众人矣?

什么都会,就等于什么都不会,因为,没有一样是精通的。

每一个都学个一知半解,那还不如不学。

也就是到了初中之后,课外兴趣班才少了一些,因为初中的课程比小学要更加紧张,作业量也更加繁重。

只是那些兴趣班虽然没有了,但课外的补习辅导班却依然不少,哪怕知道父母是为了自己能学得更多,学得更好,但苏雨晴还是不可避免地产生了抵制的情绪。

苏雨晴绘画的水平不算多高,也就是会画一些简笔画而已,素描也会一点,但只能画一些物件儿,还得是比较有棱角的那种东西,如果是画人的话,那她就根本画不像了。

涂鸦嘛,自然是偏向简笔画更多一点,几笔就能画出一样东西,而苏雨晴在本子上画的则是一只黑色的小猫,虽然简单,但还是看得出隐约和曲奇有那么几分相似。

“喵——”就在苏雨晴认真地给这只黑猫涂上黑色的时候,窗外传来了真正的猫的叫声。

曲奇紧贴着窗台站着,几乎通体黑色的它在黑夜中显得有些朦胧,只有那一双幽蓝色的瞳孔比较显眼。

曲奇虽然是一只流浪猫,但是毛发总是梳理得很干净,一点都不像一只野猫。

“喵?”苏雨晴用猫的“语言”回应道。

当然,曲奇肯定是听不懂的。

曲奇慵懒地抬起自己的前爪舔了舔,即使是站在狭窄的窗台上,也没有丝毫的紧张。

而后它用那只抬起的前爪敲了敲窗户玻璃,像是一位前来拜访的友人一般。

这只猫实在是太人性化了,苏雨晴觉得自己越来越喜欢它了,只可惜它好像不喜欢待在她家里,而是喜欢在外面四处走动。

不过看起来,它也应该没有忘记苏雨晴,时不时地还会来串个门呢……

“刷拉——”苏雨晴将窗户打开,清冷的风顿时涌入了房间里,让刚洗好澡的苏雨晴微微打了个寒颤,她现在就穿着一件宽大的衬衫和平角短裤,衣服带来的保暖效果自然差了很多。

曲奇却没有急着进来,而是叼起了窗台左角一处视觉死角里的什么东西,然后才不急不缓地走了进来,站在了苏雨晴的书桌上。

“呀!”苏雨晴看清楚了曲奇叼着的东西,惊叫了一声,忍不住站起来倒退了两步。

因为曲奇的嘴里咬着的正是一只灰色的老鼠,这只老鼠不大,看起来应该刚断奶没多久,毛色虽然是灰色的,但是并不显得脏乱,反倒是感觉如同绸缎一般柔顺。

在老鼠里,它也算是异类了吧。

苏雨晴不怕老鼠,只是怕脏而已,因为老鼠常年在下水管道和阴沟里跑,身上都是细菌……

对于她来说,老鼠的恶心程度不下于蟑螂,只是相对来说,拍死一只蟑螂不会有多大心理负担,但如果真的亲手杀死一只老鼠,就会让她觉得有些血腥了。

而曲奇就是想这么做的。

这只老鼠还没有死,但已经十分虚弱了,在之前肯定被曲奇玩弄了许久,耗尽了浑身的力气,现在除了尾巴还会稍微摆动一下,看起来就像死了一样。

如果不出意外的话,它肯定会被曲奇给吞进肚子里当作晚餐吧。

但是让人没想到的是,曲奇竟然将这只已经没有力气的老鼠放在了苏雨晴的书桌上,虽然这只老鼠才刚断奶没多久,看起来也不是很脏,但苏雨晴依然不想接近它,站在两步之外,有些疑惑地问道:“唔……?什么意思?”

曲奇听不懂苏雨晴的话,但读懂了她的表情,用猫爪子指了指这只老鼠,又指了指苏雨晴,好像在表达着什么。

“嗯?”曲奇的肢体语言实在太过简单,苏雨晴当然没法理解它的意图。

见苏雨晴还是不明白,曲奇只好咬起老鼠,又衔着它往前走了几步,放在靠近苏雨晴的地方,一只前爪摁着这只老鼠,防止它偷跑了。

苏雨晴觉得她好像有些明白曲奇的意思了。

大概……就像是一位上门做客的客人,带了一份送给她的礼物?

……
