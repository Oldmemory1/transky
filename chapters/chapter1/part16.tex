\subsection{看板娘}

早上九点的生意平平淡淡,虽然今天太阳出来了,但总体来说应该算是阴天。

只是和前几天的工作不同,因为今天苏雨晴接到了一个新的任务——帮忙宣传产品。

通俗地说,就是看板娘。

一家售卖高汤汤料的小厂家找到了这家在附近口碑和生意都还算不错的小面馆,希望能够推销一下自己的产品。

推销产品自然是要借用门口的空间,自然也不是免费的,厂家会给老板一百块钱作为租借场地的钱,并且给了老板一叠传单,希望能给每一个进来吃面的食客分发一张。

本来是没苏雨晴什么事的,但是在看到苏雨晴的第一眼,那个看起来不像业务员,倒像是工厂领导的男人一下子就看中了她。

“可以请这位小姑娘帮个忙吗?我正打算找个小姑娘来帮忙推销呢,毕竟人家更愿意看漂亮的小姑娘,而不愿意看我这个中年大叔嘛!”中年男子爽朗地笑道,“这下就正好了,也不用去其他地方找……”

“诶?”小晴的脸顿时变得通红,她有些支支吾吾地说道,“可、可……那个……我……我是男……男……”

其实苏雨晴是不想说出自己的身份的,但是知道自己身份的张阿姨和李老板都站在她身旁,与其让她们来说,还不如自己主动说出来,省得待会儿尴尬。

“哦?”那个中年男人上下打量了一下苏雨晴,摸了摸下巴,笑道,“仔细看的话,确实还是能看出有点小男孩儿的影子,长得真是清秀啊……哈,不过也没事,小帅哥也能吸引到不少人嘛。”

“那个……可是……我……我不会……推……推销……”

“男孩子这么害羞可不好啊,正好给你多锻炼锻炼嘛,这样子吧,我给你两百块钱,你就站在旁边帮我派发一下传单,介绍的事情由我来,可以吗?”

对于这种抛头露面,引人关注的事情苏雨晴其实是不太喜欢去做的,不仅是因为害羞,更是因为不太喜欢受到太多人的目光,那些复杂的目光投在苏雨晴的身上,会让她感觉浑身不自在的。

但是没办法,谁让人为财死,鸟为食亡呢?

苏雨晴很缺钱,家里还有一大堆家用品需要购置,但是手头上又没钱,昨天晚上就让她苦恼了好久。

今天,就遇上了……这样的好事……嗯……应该算是好事吧。

只是站在旁边发发传单,一天下来就能有两百块钱,要知道,苏雨晴在面馆里工作整整一个月才三百块钱呀!虽说如果算上了面馆提供的三餐肯定不止这点钱,但是还是让苏雨晴十分的心动。

她有些犹豫地看了看张阿姨和李老板,不知道该如何回答。

毕竟她现在还是这家面馆的帮工,在上班时间帮别人打零工……这不好吧?

中年男人看出了苏雨晴的犹豫,笑着对李老板说道:“没事没事,这样好了,李老板,我再给你一百块钱,借用你这个小帮工一天的时间,可以吗?”

“这怎么好意思……”

“这是当然的,生意人,就要讲究等价代换嘛。”中年男人很是阔气地先将两百块钱递给了李老板,李老板客气了一番,最终还是笑盈盈地收下了。

两百块钱,抵得上面馆生意好的时候的一天收入了呢。

去掉租金以及乱七八糟的费用,一碗面才只赚一到两块钱,一天最少也得要一百个客人才能赚到两百块钱呢。

苏雨晴也很高兴,因为这样子的话就没有后顾之忧了。

“嗯!”于是,苏雨晴也点了点头,应下了这份工作。

恍惚间,苏雨晴仿佛都看到那两百块钱在朝自己招手了。

真的不是因为苏雨晴太过财迷,而是因为没有钱的日子实在什么也做不了,有这样不用出卖肉体和灵魂,就可以轻松赚钱的工作,当然要把握住了。

在此之前,苏雨晴还得努力克服一下自己见到陌生人就容易害羞的性格。

于是,苏雨晴就站在那个比她还高几个头的大板子旁,板子上印着令人垂涎欲滴的浓汤图案,以及十分详尽的介绍。

苏雨晴的工作很简单,就是给每一个从店里离开的,以及每一个路过的人发上一张传单。

然后苏雨晴就看着那一个个路人走过,站在那里发呆。

“小伙子,该干活咯。”一旁的中年男人提醒了一句。

“啊!哦……嗯……”苏雨晴有些慌乱地点了点头,鼓起勇气将一张传单递给了一名从面馆里走出来的食客。

“您好,请看一下,我们的新产品,这种高汤佐料,只要放一点,就会让汤料非常的鲜美……”

不要误会了,这句话当然不是苏雨晴说的,而是一旁的中年男人跟在旁边说的。

那名食客看了看苏雨晴那张可爱的脸蛋,终究还是给了“女孩子”一点面子,在旁边听了中年男人一番絮叨……

不过最后还是什么都没买就走了。

苏雨晴红着脸将传单递给那些路过的人以及吃完面出来的食客,每次发传单的时候要不就低着头,要不就把头扭向别处,很多人都对这个害羞的“女孩子”有些兴趣,留下来听中年男人推销产品的也多了起来。

“呼……果然……还是不能表现得自然一点呐……”苏雨晴耷拉着脑袋,显得有些丧气,面对那些一个个陌生的客人,以及他们投来的好奇的目光,实在是让苏雨晴都不敢抬头看人了,刚一开始鼓起的勇气也早就消散得无影无踪了。

好在只是发传单问题不算很大,反正一旁有中年男人在讲解的嘛,苏雨晴的作用,大概就是一个会发传单的吉祥物吧……

被这么多人关注,也让苏雨晴感到十分的紧张,时不时地就向下压一压帽子,担心自己的平头暴露了,或者理一理衣服,生怕有哪里显得凌乱了。

每次看到有些人好奇的目光,苏雨晴就低着头看自己的脚尖。

她总是认为那些人之所以好奇,是因为她一个“女孩子”为什么还穿着男孩子的衣服吧……

女孩子穿男孩子的衣服顶多引来好奇,要是让他们知道自己是男孩子,恐怕更加好奇了吧,只不过从穿着变成了脸蛋而已……

“总是这样害羞可不好在社会上生存哦。”中年男人在百名之中还笑着说了苏雨晴一句,与其说是批评,不如说是在关心一个后辈吧……

于是,苏雨晴看准了一位从面馆里走出来的食客,再次迎了上去。

说的也是,总是这样,也不是个事儿嘛……

于是,苏雨晴努力鼓起勇气,将传单递给了那个人,第一次在发传单的时候开口说道:“那那那那个!请请……看一下!”

苏雨晴抬起头,强迫着自己看对方的眼睛。

感觉,似乎也不是那样的困难嘛。

有些事情,在迈出了第一步之后,往往会觉得其实很轻松呢。

只是,那个食客接下来的举动再一次将苏雨晴打回了原形。

这位不修边幅的年轻男人,就是昨天在面馆里吃过面,说自己是写故事的人,苏雨晴对他有些印象,所以才会表现得稍微大胆一点。

不修边幅的男人有一双温和而平静,如同止水般的双眸,他看了一眼苏雨晴,然后伸手接过了那张传单。

然后,他那粗糙的手指,轻轻地碰到了苏雨晴的指尖。

“!”苏雨晴的小脸更红了,她就像是触了电一样飞快地收回了手,之前本来能流畅地接下去介绍的话,也变得磕磕巴巴起来,“佐、佐、佐料的高……不是……汤高……不是……那个……高料……”

苏雨晴结结巴巴地说不完一句完整的话,感觉自己都快要哭出来了。

偏偏那个不修边幅的年轻男子依然平静地看着他,似乎十分有耐心的样子。

苏雨晴现在是多么希望他能对此毫不感兴趣,然后转身就走啊!

现在也不知道是该继续说,还是该停下,苏雨晴顿时陷入了进退两难的境地。

“嗯……高汤的佐料?可以直接把清水配置成高汤?”不修边幅的年轻男人终于没有再等苏雨晴介绍了,而是干脆看着传单念了起来。

“呼……”苏雨晴偷偷地松了口气,涨红着脸,向后退了几步,走回到了看板的旁边。

那个不修边幅的男人抬起头看了苏雨晴一眼,没说什么,只是走到中年男人那里买下了一大盒这样的佐料。

就连中年男人都觉得有些奇怪,这还是今天所有客人里面,最爽快的一位呢。

不修边幅的年轻男人买了佐料之后,朝苏雨晴善意地笑了笑,像是做了什么好事一样,转身离开了。

苏雨晴有些莫名其妙,不知道他的笑代表着什么意思。

难道他以为每卖出一份佐料苏雨晴就会有提成拿,所以才毫不犹豫地买了一盒吗?

唔……不会吧……

“难道他对我……有好感……?”苏雨晴感觉自己的脸都快能煮鸡蛋了,她有些扭捏地并拢着双腿轻轻地摩擦着,身体传来了些许异样的感觉。

“那个……我想去上下厕所……”苏雨晴举起手,小声地喊道。

“啊,去吧。”中年男人头也不回地应道。

苏雨晴捧着脸,害羞地朝不远处的公共厕所走去,她有些不理解为什么感觉身体会有些微微发烫,那种感觉……好微妙。

……
