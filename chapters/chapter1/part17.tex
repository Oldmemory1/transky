\subsection{日常用品采购}

苏雨晴慌忙地跑进了公共厕所里,她刚把头抬起来,就看见一个正站在小便池前撒尿的男人抬头看了她一眼。

吓得他以为有女孩子走进来了,浑身一哆嗦,尿都洒在了手上。

苏雨晴红着小脸将帽子往下压了压,飞快地窜进了一旁的隔间里,这才稍稍松了口气。

每次进卫生间都一定要到这种隔间里上,不然会感觉到尴尬得根本不知道把手往哪里放的……

因为总是去厕所,所以要总是喝水来补充缺失的水分,又因为总是喝水,所以要经常去厕所……

这几乎都快陷入一个恶性循环了,偏偏为了工作,苏雨晴不能去得太频繁,每次都是实在憋不住了才冲进厕所,饶是如此,也是几乎一个多小时就去一趟,看得那中年男人都有些诧异了。

好在发传单的工作没有疏漏,一天忙活下来,苏雨晴也出了一身汗,今天的天气相对来说还是比较温暖的呢。

春天,太阳下山的时间也要比冬天推迟一些。

下午五点,反倒是今天一天中太阳最强烈的时候,那些柔软的白云也无法遮住它最后的霞光,世间的一切仿佛都被它染上了那残阳的鲜红色……

同时,这也是苏雨晴今天一天中最开心的时候。

因为,她拿到了中年男人支付给她的工资,整整两百元钱。

在二零零四年,两百块钱可是能够买很多很多廉价的东西了,去那种二元店,可以买足足一百样商品呢。

要知道,就算是两元的商品,质量也是挺不错的呢,苏雨晴曾经的零花钱虽然也不少,但是她也经常逛两元店,因为一点点钱就可以买好多东西,对于喜欢购物的人来说,这简直是最让她满足的事情呢。

有时候运气好,还能淘到一些小精品,当然,相对地,价格也会贵一点。

你不会以为,两元店真的只卖两元的商品吧?

“小晴啊,今天也累了,就提前下班吧,明天是星期一,是你放假的日子哦。”

“唔……!对哦!今天是三月七日星期天,明天是三月八日星期一……”苏雨晴兴奋地点了点头,放假这种事情,无论是谁都会觉得喜欢的吧,“唔哇!谢谢张阿姨!”

“呵呵……遇到什么难题了就来找阿姨,一定会为你出出主意的。”

“好的~张阿姨再见~”苏雨晴高兴地朝张阿姨挥了挥手,怀揣着口袋里的两百块钱一蹦一跳地离开了,看来真的是很开心呢。

那也是当然的了,今天对于苏雨晴而言可是一个好日子呢,既得了两百块钱,又能提前下班,明天还休息一天,这种感觉,简直比放暑假还要让人觉得美妙,特别是付出自己的劳动赚来报酬这种事情……

怀里拿着两百块钱,苏雨晴的底气也足了一些,本来她今天是打算去杂货店买些便宜的东西回家的,既然有钱了,就买好一点的嘛,经久耐用也是变相的省钱呢。

超市里的促销活动也不少,苏雨晴不看别的,就专门看那些打折促销的商品——当然,得是她需要的才行。

超市里的人很多,但都是自己逛自己的,就算是聚在一起的,也都是相识的人,没有人会去刻意地关注苏雨晴,顶多是看到一个穿着男装的“女孩子”而会多看两眼而已。

苏雨晴也很喜欢超市这样的购物环境,不会有老板总是关注着自己,也不会因为空间太小而感觉有些施展不开来。

这家超市名为“大润发”,是小城市里最大的一座超市,光是超市的范围就有三层,而整幢楼则有五六层,其他的楼层不是大量餐厅聚集,就是游戏厅或者 KTV 什么的……

在这样大的超市里随意地挑选着商品,纵然人多,也不会觉得太过拥挤。

最重要的,还是能让苏雨晴彻底地放松下来,想在哪个地方待多久就待多久,而不用去考虑老板会不会不耐烦……

苏雨晴一边想着打算要买的东西,一边从头到尾地逛了过去。

首先是买了一个电热水壶,插上电就能烧热水的那种普通热水壶,外形看起来和用火烧的热水壶没有什么区别。

这种很便宜,促销价二十块钱就足够了,相对地,功能也不如那些在这个年代算是中高端产品的热水壶多——它只有一个烧热水的功能而已,而且就连烧开了都不能自己断掉电源,必须得有人拔掉插头才行。

苏雨晴家里自己用的那种电茶壶都是在烧开后会自己跳掉的呢。

有了电茶壶,当然还要有个热水瓶,热水瓶倒是不贵,虽然没有打促销的,但一个也就八块钱而已,毕竟是小城市,超市的物价也会根据所在的城市进行调整的呢。

然后苏雨晴买了一个淡粉色碎花窗帘,这个就贵了,要三十五块钱,主要是因为它还带一个可以固定在墙上的支架,这让窗帘即使不挂在窗户上,也可以挂在别的地方——只要能面墙固定住就可以。

同样是一个没有促销的商品。

足足三十五块钱呐!这可相当于那两百块钱的七分之一了,虽然很肉痛,但是一想到挂上窗帘后自己的卧室会看起来更加舒服一些,苏雨晴还是咬了咬牙,买了下来。

只要有能力,苏雨晴都希望把事情做得完美一些呢……

从某种意义上来说,像她这样的人,大多数都是完美主义者吧,不然也不会因为身体和自己所想的理念不同而走上这样的道路呢……

除此之外,还有一个小小的晾衣杆,一组衣架,还有洗澡的香皂,洗衣服的肥皂,板刷……

各种各样上一次因为钱不够而没有买的日常用品,加一起花了一百块钱,总算买了个齐,不用再担心生活中缺这样缺那样了,因为最基本的东西都有了嘛。

苏雨晴在心中计算着每一样商品的价格,然后再思考着自己还剩下多少钱,还有什么东西需要买……

购物有时候真的会让人上瘾呢,虽然已经把需要的东西都拿了,但是苏雨晴还是忍不住想要在这大超市里再逛一会儿。

她推着购物车悄悄地走进了服装区,有些心虚地瞄了几眼那些挂在衣架上的女装,生怕被那些热情的促销员发现了。

虽然她戴着帽子,一般人看不出她是男孩子,但是……万一呢……那不就尴尬了嘛?

而且超市的服装区实在是挑不好什么衣服,价格也相对来说比较贵,买衣服,要又实惠、质量又好、穿起来也好看的话,最好还是去那种服装批发市场之类的地方更好呢。

倒是有几件春天的新款连衣裙看起来挺漂亮的样子,只是上面的价格实在是让苏雨晴望而却步,随便一件就要两三百块钱,实在不是现在的苏雨晴能够负担得起的。

苏雨晴只好有些失望地离开了服装区,然后……她站在玩具区里走不动路了。

十五岁的年纪,也算是一个大孩子了,但终究还是一个孩子,对于有些玩具总是充满了喜爱的,而且还总是忍不住诱惑……

之所以成年人不会再受到玩具的诱惑,那其实是有对于他们而言更值得去追寻的东西在诱惑着他们呢!

那么,苏雨晴到底是被什么东西给迷住了呢?

嗯……一盒国产的芭比娃娃套装。

虽说那种国产的娃娃做工实在是特别劣质粗糙,但胜在价格低廉、种类丰富,还附带各种赠品。

而且这盒玩具上挂着的大大的“促销”标签,更是让苏雨晴无法移开目光了。

原价两百三十元,现在清仓价,只要六十五……

巨大的折扣让苏雨晴更加心动了,或许过了这个村,就没这个店了……

看这盒芭比娃娃的做工,在国产货中确实算是上乘了,还搭配了三套衣服以及各种各样的迷你小家具……

“唔……”苏雨晴有些害羞地看了看四周,忍不住伸出手摸了摸那个大包装盒,十五岁虽然还是孩子,但是已经会因为没玩具而感到有些羞耻了,因为这个年龄还买玩具,在很多人眼里,都可以扣上“幼稚”两个字了呢……

但是苏雨晴实在是经受不住这样的诱惑,要是没钱也就算了,偏偏她的口袋里有足够的钱可以买下这盒国产版的“芭比娃娃”……

就像是普通的男孩子喜欢玩“变形金刚”或者“赛车”之类的玩具一样,像苏雨晴这样的有着一颗女孩子内心的男孩子,也喜欢这种可以用来扮家家酒的玩具,特别是那种能换装的人偶……简直是抓住了她的心!

最后,苏雨晴还是拿起了那盒清仓的芭比娃娃玩具,放进了购物车里,为了掩饰尴尬,她还故意将它放在最底下,这样子别人就不会一下子看到了。

嗯……有一种掩耳盗铃的感觉呢。

苏雨晴计算着剩余的钱,又去楼下的散装零食区称了一些小零食,可以在平时在家里无聊的时候吃上几个,除了那种“可爱”的玩具之外,零食,也是女孩子的最爱。

而且,还不仅限于甜食,苏雨晴对于豆腐干以及鸡爪什么的都是特别喜欢的呢。

一颗有着近乎纯粹的少女心的男孩子,这也是为什么要称呼她,而不称呼他的原因呢。

……
