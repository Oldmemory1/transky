\subsection{失眠}

夜色愈发的阴沉了,月光和星光全被雾云遮住了,附近的街道也愈发的寂静,偶尔能听到野猫和野狗翻捡着垃圾桶的垃圾时所发出的声音,但却没有人的声音。

也是,这么晚了,还有几个人会在外面。

但是此刻苏雨晴却是多么的希望外面的街道能热闹一些,听到那种嘈杂的声音,或许会让她感觉安心一些。

大脑迷迷糊糊的,想要睡,却怎么也睡不着,想要醒,却难以清醒。

这种感觉比一般的失眠还要难受。

“啪哒。”一滴漆黑如墨的雨点落在了地上,就像是冲锋的信号一般,数之不尽的墨色雨水倾泻而下。

雨并不是很大,没有发出炒豆子般的声响,轻轻柔柔的,看起来连绵不绝的样子,估计这场雨会下很久吧。

闹钟响了起来,虽然外面还是一片漆黑,但时间却已经是早上五点二十分了。

苏雨晴感觉脑袋昏昏沉沉的,很费劲地坐在了床上,朝窗外望去。

窗外的雨点是线状的,看起来就像是一幅水墨画,有一种奇妙的美感。

但是此时的苏雨晴却没有那份心情欣赏这份美,她一晚上都没有睡好,但是现在却要去上班了,为了自己的工资,为了养活自己,她只能起床去上班了。

“嘶——”苏雨晴刚把脚放在地上,顿时就感觉一阵钻心的痛,昨天被戳破的水泡还没有愈合,走在地上的疼感比有水泡的时候还要强烈。

苏雨晴忍着痛,一瘸一拐地走到了卫生间里,因为浑身都提不起劲,而且脚底还传来阵阵的疼痛,所以苏雨晴的洗漱也就简单了许多。

就像是浑身的骨头被人抽了出来一样的感觉,有时候不扶着墙感觉都有点站不稳。

“唔……”苏雨晴紧紧地皱着眉头,一脸痛苦的表情,她半眯着眼睛,扶着墙,小心翼翼地下了楼。

雨下得虽然不大,但也不小。

苏雨晴没有伞,只能淋雨过去,想要少淋点雨,就只能跑着过去,但此时的苏雨晴又哪里跑得动呢,顶多是勉强还能走路,她只能尽量选择有屋檐的地方,紧挨着墙慢慢地挪动着脚步。

身体的难受让苏雨晴甚至无法集中精力思考,每一次想要转移注意力,都会被疼痛的感觉和心口堵塞的感觉给打断。

“呼……呼……”苏雨晴在一处屋檐下蹲了下来,些许雨水被风吹着斜落在她的身上。

才走了没几步路,她就感觉一点力气都没有了,胸口闷得难受,似乎快要喘不过气来了。

“难道是药量太大了吗……”

如果初次用药,一次性用的药量过大,可能会对心脏造成负担。

而且副作用非常大,随着时间的推移,几次后,等身体渐渐适应了才会好转过来。

就这样走走停停,苏雨晴总算是好不容易走到了面馆里,虽然尽量在屋檐下走,但她身上的衣服还是湿了不少,被风一吹,顿时让她感觉浑身冰冷,就像是被冰块冻住了一样。

虽然天很黑,还下着雨,但是老板还是早早地来开了门,漆黑的街道中,面馆里亮着的灯是那样的明亮,又是那样的微弱……

张阿姨和李叔叔已经开始忙碌了起来。

“小晴,来啦。”张阿姨朝苏雨晴打招呼道。

“嗯……”苏雨晴捂着胸口,勉强挤出一个笑容,慢慢地坐在了椅子上,这样会让她感觉舒服一些。

“小晴,你怎么了?脸都发白了?”

“我……没事……”

“是不是没睡好?要不再睡一会儿吧?反正今天下雨,客人也不会太多。”

“小晴,瞧你柔柔弱弱的,叔叔给你放个鸡腿补补吧,主食的话,炒面怎么样?”李叔叔大声说道。

“嗯……”苏雨晴虚弱地点了点头,正了正脑袋上的帽子,直接趴在了桌上,之前明明还怎么样都睡不着的她,趴在桌上后却迅速地进入了梦乡。

苏雨晴第一次觉得睡觉原来是那样舒服的事情。

只要能好好地睡上一觉,就算是神仙也不换呐……

“唰啦——唰啦——”厨房里飘出炒面的香味,李老板端着一碗炒面走了出来,对张阿姨说道,“老婆,拿个卤鸡腿出来,看小晴累的。”

“人家八成是穷人家出身,身子骨虚也是正常的,是该给她补补。”张阿姨点了点头,拿出一个最大的鸡腿放在炒面的上面。

李老板则端着这份加了鸡腿的炒面走到了苏雨晴的身旁。

浓浓的食物的香味都没能让苏雨晴醒来。

“小晴?小晴?”李老板轻轻地推了推苏雨晴的身子,后者却没有什么反应,看起来睡得很沉的样子。

“吃早饭了小晴。”张阿姨也帮忙喊道。

“唔……嗯……”苏雨晴有些困倦地揉了揉眼睛,看到了放在自己面前的一盘色香味俱全的炒面,但偏偏没有一点食欲。

“小晴,先吃点东西吧,不然身体会更难受的。”张阿姨说道。

“嗯……”苏雨晴轻轻地点了点头,勉强撑起自己的身子,拿起一双一次性筷子,慢慢地吃了起来。

就算是吃面的时候,她都有些半梦半醒的样子,眯着眼睛,像是在梦游。

食物的味道怎么样?苏雨晴不知道,她只感觉有东西被自己咬进了嘴里,然后下意识地嚼碎咽下去,就算是如同嚼蜡都好歹有个蜡的味道,而苏雨晴此时却感觉像是在咀嚼空气。

“咣当。”实在太困了的苏雨晴终于撑不住再次趴倒在了桌上,筷子和只吃了小半个的鸡腿都掉在了地上。

“又睡着了?”李老板问。

“看来真的是太累太困了。”张阿姨也轻轻地摇了摇头,“算了,让他睡吧。”

“嗯,反正今天也不会忙。”李老板点了点头,将桌子收拾干净,把苏雨晴抱到了最里面的靠墙的椅子上。

苏雨晴一点感觉都没有,换了位置之后依然保持着之前的姿势趴在桌上睡觉,那架势就像是几天几夜没合眼了似的。

……

苏雨晴感觉自己就像是坠入了深渊,而且这个深渊还很深,好像没有底一样,她就这样一直下坠着,四周都是一片漆黑,什么也看不见,什么也感觉不到。

终于,下坠的势头止住了,四周的空间开始变得亮堂了起来,其实只是冒出了一小团微弱的火光而已。

苏雨晴感觉自己好像站在一座公园的入口,身后是一片漆黑,只有公园里有着昏暗而明灭不定的光。

她看了看身后,又看了看身前,犹豫了一会儿,最后还是一脚迈入了里面。

这是一个荒芜而破旧的公园,一旁的树木因为没有人来修剪,所以长得十分肆意,奇形怪状的,有些甚至像是张牙舞爪的恶鬼。

杂草也早已攻陷了公园的小路,苏雨晴就走在这能够没过她脚踝的杂草中,漫无目的地走着。

路灯还在散发着昏暗的橘红色光芒,而路灯的灯柱却已经锈迹斑斑,有些甚至能够看到那暴露在空气中的线路。

公园的健身器材也早已破的破,坏的坏:少了一边的步行器;缺了口子的太极环;没有托盘的扭腰器;线都烂光的拉伸器……

“吱呀——吱呀——”安静的公园里突然传出了这有些刺耳的声音,让苏雨晴有些毛骨悚然。

“吱呀——吱呀——”

苏雨晴有些害怕地看着四周,生怕在什么地方有可怕的东西突然窜出来将她吞噬。

“吱呀——吱呀——”刺耳的声音依然在继续着,似乎没有停下来的意思。

苏雨晴害怕地感觉浑身都有些颤抖,但是好奇心还是驱使着她小心翼翼地朝发出声音的方向走去。

“吱呀——吱呀——”

声音越来越近,也越来越清晰了,隔着一棵遮挡着视线的大树,苏雨晴看到有一个影子在上下晃动着,看起来像是秋千。

只是被风吹起来的秋千而已吗?

苏雨晴壮着胆子绕过那棵树,走到了秋千前,一个长相清秀的小男孩儿正坐在秋千上摇晃着,眉目间似乎和苏雨晴有些许相似。

秋千的影子依然在上下晃动着,但是……这个小男孩儿,却没有他的影子。

苏雨晴站在了原地,不敢向前挪动半步,未知的东西,才是最让人恐惧的。

“小姐姐,你在干嘛呀?”小男孩儿笑着开口了,他的声音很好听,笑声也很天真,但看在苏雨晴的眼里,却是那样的诡异而恐怖。

“没……没做什么……”

“小姐姐知道离开的路吗?”

“离开的路……?”

“是呀,我被困在这座公园里好久了,一直找不到离开的路呢。”小男孩儿仰起脑袋,看着苏雨晴,一脸天真地问道,“小姐姐知道吗?我真的好想离开这里呐。”

“……我……我不知道离开的路……”

“那小姐姐为什么要来这里呢,是要来找什么吗?”

“找什么……好像……是要找些什么,但是……是要找什么呢……”苏雨晴有些头疼,无论如何都想不起来自己为什么会出现在这里,是要找些什么东西。

四周的路灯开始变得明灭不定起来,这公园的一切,包括那个小男孩都在缓缓地虚化。

小男孩朝着苏雨晴笑道:“小姐姐,有机会再见面哦,希望下次你能带我离开这里……”

一切,都消失了。

世界,再次陷入了一片漆黑。

……
