\subsection{流落街头}

天空彻底地暗了下来,夜晚的小城市虽然不如杭州那样繁华,但也是灯红酒绿的,这里是火车站附近,到了夜晚,饭店的生意也愈发的活络起来,到处都是络绎不绝的客人。

苏雨晴漫无目的地在街道上走着,她的双眼没有焦距,不知道自己将要去哪里,也不知道自己要去做什么。

就这样不断地向前走着。

她走了很久,四周的店也越来越少,就连那些在路边摆摊的,也差不多要准备收摊了。

时间,大概已经是午夜了吧。

苏雨晴感觉自己就像是被抽走了灵魂一样,虽然双脚已经麻木,但还是在向前走。

一直走到双腿发软,都快撑不住自己的身体了,苏雨晴才缓缓地停了下来。

现在已经很晚了吧,四周的街道上除了一家二十四小时营业的便利店外,已经没有其他的店还开着了。

阴云依然遮挡着天空,无论是月光还是星光,都无法倾泻下来。

除了昏暗的灯光之外,就没有其他什么还发光的东西了。

明明是春天,却让苏雨晴有了一种荒凉而萧瑟的感觉。

她疲惫地拖着行李箱坐在了路边,感觉到无穷的迷茫和彷徨。

冰冷的风轻轻地吹过,让苏雨晴不由得抱紧了身体,也让她已经有些麻木的神经重新缓过劲来。

没有钱,寸步难行,但最起码,苏雨晴还活着,还活着,就还有希望,钱可以再赚,但如果就这样自暴自弃的话,真的有可能会饿死在街头的……

或许只会成为某一天的报纸头条,随着时间的消逝被人们忘得一干二净而已。

“如果这是命运,那我就偏要反抗,偏不能让它如愿!”苏雨晴轻轻地咬住了嘴唇,攥紧了拳头,对着那漆黑的夜空大声说道。

寂静的夜中,若隐若无地回荡着苏雨晴的这声大喊,像是在坚定着她的信心。

是呀,既然决定成为一个女孩子,既然决定走上这条忤逆“上帝”的路,那在最开始的时候就已经决定和命运抗争到底了,现在只不过是命运制造的一个小小的挫折而已,又算得了什么呢?

孩子的情绪是不稳定的,她们很容易就悲伤,很容易就彷徨,也很容易重新变得坚定,重新变得坚强……

苏雨晴,也就正处在这样的年岁里。

远处是一座小型的加油站,这里是一条长长的公路,前面有一个没有人等候的公交车站,后面有一座看起来荒废了很久的小木屋。

苏雨晴拖着疲惫的身体,想要走到小木屋里度过这个晚上,但遗憾的是小木屋的门紧锁着,尽管门锁已经生锈了,但苏雨晴还是无法将它打开。

“呼……算了……”苏雨晴浑身都使不上力气,站都快站不住了,更别说去开这样一道门了,她很快就放弃了,干脆地坐在了小木屋旁。

夜已经很深了,就算是想要找包吃包做的工作也不是这个时候,而要住旅馆的话,苏雨晴的身上又没有钱。

所以只能将就地休息一晚了,等到明天太阳升起的时候,再寻找出路吧。

人类是脆弱的,脆弱到一件小事就能让人心理崩溃;人类也是坚强的,坚强到即使陷入困境,也要咬着牙,努力地活下去。

“希望”,是能让人类继续坚强活下去的火种,想着那日后的种种美好,总会让人觉得安心许多。

纵然身上已经一分钱都没有了,苏雨晴也依然想着,明天将会更好,盼望着第二天的太阳升起,盼望着找到一份合适的工作……

“好饿……”

苏雨晴拨弄着木屋墙壁上的青苔,揉了揉瘪下去的肚子,胃在不断地抗议着,身体也因为缺少卡路里而难以恢复力气。

如果青苔可以吃的话,苏雨晴一定会毫不犹豫地吞下一大把吧。

“对了……还有吃的……”

就在苏雨晴难以忍受饥饿的时候,突然想起来了中午吃剩下的嵌糕,她只吃了一半,还有一半呢,用来填饱肚子显然是足够了。

一想到嵌糕里丰富的馅料,苏雨晴就觉得肚子更饿了。

嵌糕早就已经冷掉了,而且因为是年糕做的,所以在冷掉之后还会变得特别的硬,硬得像是要把人的牙齿都给硌掉一般。

但是对于饥肠辘辘的苏雨晴而言,这就是上天最好的恩赐。

或许这就是所谓的,天无绝人之路吧。

苏雨晴一点一点,小口小口地嚼着嵌糕,她咬下一小块,放在嘴里让它变得软一些后,才细细地品嚼,似乎是想要将那种香味深深地印刻在自己的脑海里。

对于饥饿的人而言,任何食物都是美味的。

其实以前苏雨晴是不太喜欢吃胡萝卜的,但是今天她却觉得嵌糕里那冷掉的胡萝卜丝简直就是人间的佳肴,她还从来没有吃过那么好吃的胡萝卜呢……

挫折和磨难,真的能让人改变许多呢。

苏雨晴吃完了那半个嵌糕,还觉得有些意犹未尽,但很快她就觉得肚子胀得有些难受了,只是这种胀肚的感觉,实在是比饿肚子的感觉要好受得多呢。

最起码,现在即使肚子胀胀的,也会让苏雨晴有少许的幸福感呢。

有时候,幸福,其实就是那么简单的事情。

春天的夜晚很冷。

苏雨晴抱着自己的书包,侧躺在行李箱上,蜷缩着身子,尽量让自己暖和一些,但是那些时不时吹过的冷风,却总是将她那可怜的热量带走。

“好……好冷……”苏雨晴的嘴唇有些发紫,手脚也有些冰凉,在这样的天气里睡一个晚上,恐怕第二天起来就要感冒了吧。

此时的苏雨晴才觉得家里那柔软的席梦思床是那样的舒服,是那样的温暖……

好想有个家,有个温暖的家,家不用太大,不漏风,不漏雨,有一张铺上了柔软棉被的床,就好……

苏雨晴在心中默默地想着,还是抵挡不住身体和心灵上的双重疲倦,合上了眼睛,昏昏沉沉地睡了过去。

这一觉睡得并不安稳,几乎每隔一个小时苏雨晴就要被冻醒一次,她只能稍稍活动一下身子,让自己暖和一些,才能继续入睡。

“轰——嗖——”公路上偶尔会开过一辆疾驰的轿车,给这安静的公路添上些许沉闷的“鼓点”。

公路旁其实并非没有任何的声音,有风吹树叶的声音;也有野性的耗子鬼鬼祟祟地从下水道里钻出来觅食的声音;还有野猫们“凄惨”的仿佛婴儿啼哭般的叫声……

苏雨晴断断续续地做着一个梦。

梦见她在外面实在生活不下去了,回到了家里,却被父母赶了出来,那个本应该温馨的家,却变得那样的冰冷,曾经慈祥的父母,也变得那样的陌生。

梦中的她重新开始了流浪,却不知怎么的暴露了自己的身份,街道上的行人都对她指指点点的,甚至有人对她公然侮辱,那些行人们却只在一旁看笑话,甚至,笑得更加开心了。

后来,苏雨晴被人贩子拐走,将值钱的器官卖掉后,砍去了四肢,变成了街边乞讨的小乞丐,每天为了那一顿微薄的食物,而忍受着身体和心灵上的双重折磨……

而她的父母,在街上见到她后,不仅没有可怜她,反倒将她乞讨的碗踢翻,大肆地嘲笑她……

梦,醒了。

苏雨晴的脸上挂着两行清泪,她捂着胸口,心脏在缓缓地跳动着。

虽然只是一场梦,但为何却觉得那份钻心的痛,是那样的真实,是那样的……刻骨铭心呢?

苏雨晴抱紧了身子,就像是一只柔弱无力,没有依靠的小猫一样,害怕着、胆怯着、瑟瑟发抖着。

她不想那样,不想过上那样的生活,她也不想回头,她要继续向前……继续向着自己的梦想而前进。

性别,是上帝赐予的,她无法选择,但是,凭什么,她不能在后天抗争呢,她凭什么没有选择自己性别的权利呢?

只是因为那些人伦道德的约束?

苏雨晴睁开了眼睛,她的双眸很清澈,就像是没有被污染过的清泉一般。

看起来柔柔弱弱的,但是在那份柔弱之下,却有着不屈和倔强。

她不服,所以,要抗争。

笼罩在天空中的阴云不知道在什么时候消散了,清澈的夜空中,没有一丝的云彩,清冷的月亮高挂在空中,将皎洁的月光洒在苏雨晴的身上。

小城市并不发达,也正是因为不发达,在这座城市里,还能看到那漫天的繁星,那随意挥洒的漫天星光。

看着那天空中的月亮和数不尽的星辰,苏雨晴心中的那份寞落也减轻了许多。

无论如何,还有它在陪着我,它不会因为自己的身份,不会因为自己的性别而冷落自己,它的光,总是在,总是指引着她……

无论是月亮,还是太阳,亦或是星光。

现在已经是凌晨了,月亮也将要落下了,东边的地平线上探出一抹金光,将那一小片天空都染成了金色。

或许过不了多久,天就会大亮吧。

看起来,今天应该会是个好天气呢。

这一晚上虽然没有休息好,甚至还有些着凉,但是此刻的苏雨晴却觉得精神格外的振奋,新的一天到来了,这也意味着新的希望。

在出发之前,苏雨晴决定再翻找一遍书包和行李箱,说不定会找到可能被自己塞到哪个角落里的钱包呢?

……
