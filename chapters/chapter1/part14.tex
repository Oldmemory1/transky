\subsection{苏雨晴的日记}

时间就这样缓缓地流逝着。

苏雨晴突然觉得这样下雨的天气也不错,因为客人很少,所以可以让她有大量的时间发呆和休息。

很快,就到了晚上九点。

雨,还在下着,只是从倾盆大雨变成了淅淅沥沥的小雨。

天空很黑,没有哪怕一丁点的月光和星光照射下来,因为那密集的云层将它们全都遮挡住了。

“小晴,店里有一把雨伞,你拿去用吧。”离开店之前,张阿姨叫住了苏雨晴,说道。

“不用了……雨下得也不是很大,我直接走回去就可以了。”

“撑着吧,我们有伞,少淋点雨,这个天气可是很容易感冒发烧的呢。”

“唔……那就谢谢了……”

苏雨晴接过伞,走到店外撑开,一只手拿着伞,一只手拿着装满了水的杯子,没入了那漆黑的夜幕中。

“明天见。”张阿姨挥手说道。

“嗯,明天见。”苏雨晴扭过头笑了笑,这才真正地转身离去。

回到苏雨晴自己布置的温馨的家中,感觉心情都放松了许多。

到家的第一件事就是先洗漱,苏雨晴一边用清冷的水打湿自己的脸蛋,一边看着镜子中的自己。

没有什么变化,这是当然的,如果药效有那么快的话,代价恐怕就是瞬间致死吧。

效果越好,副作用就越大,大多数药都是如此的。

变化其实还是有的,那就是额头上又冒出了几个痘痘,大有让整个额头都长满痘痘的趋势。

痘痘不能随便挤,不然很有可能会留下痘印,苏雨晴能做的只是尽量把脸洗干净,让污垢不要堵塞在毛孔里而已。

只要处理得好,痘痘很快就能消退的。

或许是因为今天一天基本都是坐着的缘故,脚底的水泡伤口已经结了痂,走在路上也不会感觉有多疼了,只是稍微觉得有些痒而已。

即使是洗脚,苏雨晴也是十分认真的,她将脚放在洗脚的小脸盆里,仔细地搓洗着每一块地方,包括脚趾之间的缝隙。

等到脚被搓洗得看起来有些微微发红的时候就可以了。

不过冰冷的水还是让苏雨晴觉得有些受不了,所以还是稍微懈怠了一些。

“姆……看起来真的应该买一个热水壶呢……不然都没法烧热水……”苏雨晴抿着嘴想着,将脚擦干,走到了自己的“卧室”里。

苏雨晴写日记并非是每天写,只有在想写的时候才写,或者是隔了一个星期没写的话,才会去写上一次。

在写日记之前,苏雨晴先拿出了一个小本子,小本子很可爱,外面印着桃花的图案,这是苏雨晴上个月时参加学校里的绘画比赛时所获得的,还没有在上面写过一个字,现在则被苏雨晴作为备忘录和账本来用。

苏雨晴在备忘录上写下了“热水壶”、“热水瓶”、“窗帘”这几个词语,然后就将它合上,打开了自己的日记本。

苏雨晴的笔停在日记本上,思考了良久,又把它放到了一旁。

只因为这本日记中记录着曾经的事情。

既然已经独自一人来到小城市生活了,那就意味着新的开始,以前的一切都放下吧,保存好,作为自己曾经的回忆就足矣。

新的开始,自然要用新的日记本,好在苏雨晴带的新本子够多,她挑了一本封面画着一个昏暗的路灯的胶装日记本,放在了桌上。

至于原来的那本,则被她放回了抽屉里。

苏雨晴在日记本上写下了大大的“日记”两个字,还用心地在这两个字旁边画上了许多可爱的小图案,日记本的第一页被装点得很漂亮,有一种淡淡的清新的感觉。

看着自己完成的绘画,苏雨晴满意地点了点头,将日记向下翻了一页,然后用力地压了压,让它在写的时候不至于总是翻回来。

“二零零四年,三月六日……”苏雨晴一边小声地默念着,一边在书页的定格写上了文字。

苏雨晴的字体很娟秀,根本就不像是男孩子的字体,而且她写字的力道都很轻,让她的字体看上去也如她的人一样柔柔弱弱的。

“昨天,正式开始吃了药,补佳乐两粒,螺内酯六粒……”

“晚上睡不着,胸口很难受。”

“早上很困……睡醒了之后感觉舒服多了……”

“生活已经开始稳定了下来,我现在已经有了自己的小窝和一份稳定的工作……”

“上午时那个男人所说的话都好有哲理,不知道他到底有着怎么样的故事呢?”

“我觉得这样的生活,其实挺好的……”

这篇日记是苏雨晴新的开始后的第一篇日记,要写的东西有很多,那些感谢、那些生活中的琐碎……

苏雨晴足足写了三页纸才停下笔,看着自己写得满满当当的日记,心中也有着些许的满足。

“未来的我,每天都过得开心吗?”最后,苏雨晴在末尾加上了这句话,才停下笔,轻轻地伸了个懒腰。

窗外黑漆漆的,在楼下只有一盏昏暗的路灯在照明。

和日记本封面上的画竟然有着些许的相似。

或许画中如果再画上雨的话,会显得更加凄美一些吧。

大概是和本身的性格有关,苏雨晴对于这种伤感的事物总是特别的敏感,她就是属于那种见到落花也能为其轻叹的人吧。

“呼……睡觉吧……”苏雨晴从椅子上站了起来,对着那窗外漆黑的世界扮了一个可爱的鬼脸,“呀,对了,药还没吃呢……”

就着水将药吃完后,苏雨晴再想了一遍有什么事情没做,确定没有了之后,才安心地钻进了被窝里。

“咔嗒。”灯,被关上了。

可能是因为今天白天睡了很长时间的缘故,也有可能是药物所造成的副作用,苏雨晴感觉不到疲惫,在床上翻来覆去地有些睡不着,胸口又有些发闷了,但比昨天晚上要好一些,饶是如此,还是辗转反侧地一直到半夜里才睡着。

窗外的风轻轻地吹着,将那怎么样都不愿意停下来的小雨吹得有些偏斜。

一只浑身漆黑的流浪猫融入在夜色中,从墙头窜了下去,似乎是打算溜进谁家的院子里找些食物。

它优雅而轻缓地走着,就像是一名独行的刺客一般,处变不惊,沉稳冷静。

黑猫淋着雨,走到了院子里的一扇窗户前,敏捷地跳了进去,没有发出丝毫的声音。

一会儿之后,这只黑猫刺客叼着一块被人类丢弃在垃圾桶里的鱼头窜了出来,再次敏捷地翻出墙头,消失在这夜幕中,或许去哪里享受它美味的夜宵了吧。

……

还是那个寂静的诡异的公园,耳边还是传来那唯一的声音——秋千晃动的声音。

苏雨晴走在这座公园里,循着记忆找到了秋千所在的位置。

但此刻的秋千上却并没有坐着人,秋千只是被风吹着,上下地晃动着。

“吱呀——吱呀——”

诡异的声响让即使知道这只是梦的苏雨晴也感到十分的害怕。

或者说,在梦中,人类其实是很难控制住自己的情绪的。

在梦中所表现出来的,往往都是最真实的自己。

苏雨晴四下张望着,希望能找到那个有些诡异的小男孩儿。

“小姐姐,你是在找我吗?”突然,一个声音在苏雨晴的背后响起,苏雨晴吓得立刻转过身,倒退了好几步,紧张地看着这个小男孩儿,而他现在所做的事情更是让苏雨晴感到毛骨悚然。

只见他手里捧着一堆拇指大的小石头,然后拿起一块放进嘴里,“咔嚓咔嚓”地吃了起来,像是咬碎骨头所发出的声音。

“你……你……你在吃什么……”

“好吃的,小姐姐要吗?”小男孩儿拿起一块看起来最漂亮的石头,伸向苏雨晴,问道。

“我……我不用……”苏雨晴用力地摇了摇头,道,“这可是石头……”

“嗯,是石头呀。”小男孩儿的回答倒是很正常,苏雨晴原本以为他会说这其实是看起来像石头的食物呢。

“你知道还吃?石头吃进肚子里,肯定会很难受吧?”

小男孩儿皱了皱眉头,轻轻地点了点头,道:“是呀……小姐姐,吃石头真的很难受呢……但是对于我来说,它很重要,如果我不吃石头的话,我可能会活不下去……但是我吃了它的话,却会很难受……而且一次比一次更难受……”

就像是慢性自杀一样。

“既然这么难受,为什么还要吃?”

小男孩儿摇了摇头,又将一块石头放进嘴里,道:“小姐姐不会明白的,我必须得吃石头,只有吃石头,才会让我觉得安心,只有吃了石头,才会像我的梦想更进一步……”

苏雨晴睁大了眼睛,有些无法理解小男孩儿所说的话。

“小姐姐以后就会明白了。”小男孩儿朝苏雨晴露出一个温和的笑容,走到了秋千旁,坐了上去,一边摇晃着,一边吃着石头,“小姐姐,你还在找东西吗?找到了没有?”

“找东西……找什么东西……”苏雨晴的头突然就觉得很疼,“我……我想不起来……到底是在找什么东西……”

她抱着脑袋蹲在地上,一副痛苦的模样。

四周的路灯发出“兹拉兹拉”的声响,开始变得明灭不定起来。

“唔,小姐姐,看来我们又要分开了呢,下次,再见面吧……”

一切都消失了,苏雨晴的梦境,再次陷入了一片漆黑。

……
